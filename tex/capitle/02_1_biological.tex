\subsection{Biological Background} \label{subsec:biological_background}
To have a better understanding of our project, we need to have a closer look at some biological topics and concepts.\\

% Cancer overview
\textbf{Cancer} is a complex, multifactorial disease characterized by uncontrolled cell growth.
Research shows that widespread metastases are the primary cause of death from cancer~\cite{who_cancer_fact_sheet}
and that cancer comes in more than 100 different types~\cite{nci_cancer_types}.
The five most common forms of cancer worldwide in 2022 were lung cancer ($>2.4$ million),
breast cancer ($>2.2$ million), colorectal cancer ($>1.9$ million), prostate cancer ($>1.4$ million) and
stomach cancer ($>0.9$ million).
In 2022, cancer was one of the leading causes of death globally,
with approximately 10 million fatalities~\cite{ferlay2024global}.
It can be caused by a combination of genetic and environmental factors,
such as diet, radiation, age, exposure to certain chemicals or viruses~\cite{nci_cancer_risk}.
Early detection and the right treatment can significantly improve the chances of curing many types of cancer.
\\

% lung cancer
In this study we will focus on \textbf{Lung cancer}, a type of cancer that affects the lung organ.
There are two subtypes: non-small cell carcinoma (NSCLC) and small cell carcinoma (SCLC)~\cite{nci_lung_cancer_types}.
The primary risk factor for developing lung cancer is smoking,
but passive smoking and environmental pollution also significantly increase the risk~\cite{nci_lung_cancer_types}.

Symptoms of lung cancer often resemble those of common colds or other minor illnesses, such as coughing and fatigue.
This makes it challenging for patients to receive timely treatment, often leading to a diagnosis at an advanced stage~\cite{who_lung_cancer}.
\\

% gene expression
\textbf{Gene expression} is a crucial factor in the investigation of cancer.
It describes the process of transcribing DNA into RNA molecules that code for proteins,
which are then translated and regulated through complex interactions to control the production
and function of gene products~\cite{nhi_gene_expression}.
Alterations in gene expression can contribute to the uncontrolled multiplication and abnormal behavior of cancer cells,
which ultimately leads to the development and progression of cancer~\cite{ferlier2022regulation}.
\\

%TPM
To accurately compare gene expression data across different experiments, it is essential to normalize and quantify the expression levels.
The most commonly used metrics for this purpose include Reads per million mapped reads (RPM),
Reads Per Kilobase Million (RPKM), Fragments Per Kilobase Million (FPKM),
and Transcripts Per Kilobase Million (TPM)~\cite{cd_geneomics_gene_expression}.
Among these measures, \textbf{TPM} is widely adopted in available datasets.
Therefore, we will utilize it as our metric of choice for quantifying gene expression levels.
\\

% Gene IDs
A common way to identify genes is through the use of unique identifiers known as \textbf{Ensemble IDs}.
These are unique ids for genes, proteins and other genetic elements collected in the Ensemble database
from 1999 by the European Bioinformatics Institute and the Wellcome Trust Sanger Institute~\cite{ensembl_project}.
These make it easier to compare and analyze gene data across different datasets.
\\

% Multiomics data
Omics is a collection of fields in biology that end in -omics, such as genomics, proteomics, metabolomics, etc.~\cite{subedi2022omics}.
\textbf{Multi-omics data}, the combination of data from different omics fields,
are used to gain a better understanding of associations between biological molecules (e.g., genes and proteins)
and their interactions~\cite{Subramanian2020MultiomicsDI}.
Since we will be using genes and proteins in our study, we also have a multi-omics approach.
\\
\subsection{Biological background} \label{subsec:biological_background}
% Cancer overview
\textbf{Cancer} is a group of diseases characterized by uncontrolled cell growth.
According to research, widespread metastases are the primary cause of death from cancer\cite{who_cancer_fact_sheet}.
There are over 100 different types of cancer\cite{nci_cancer_types}.
The five most common forms of cancer worldwide in 2022 were lung cancer ($>2.4$ million),
breast cancer ($>2.2$ million), colorectal cancer ($>1.9$ million), prostate cancer ($>1.4$ million) and
stomach cancer ($>0.9$ million).
In 2022, cancer was one of the leading causes of death globally,
with approximately 10 million fatalities\cite{ferlay2024global}.

It can be caused by a combination of genetic and environmental factors,
such as diet, radiation, age, exposure to certain chemicals or viruses\cite{nci_cancer_risk}.
Early detection and the right treatment can significantly improve the chances of curing many types of cancer.
\\

% lung cancer
\textbf{Lung cancer} is a type of cancer that affects the lung organ.
There are two main subtypes: non-small cell carcinoma (NSCLC) and small cell carcinoma (SCLC)\cite{nci_lung_cancer_types}.
The primary risk factor for developing lung cancer is smoking,
but passive smoking and environmental pollution also significantly increase the risk.

Unfortunately, symptoms of lung cancer often resemble those of common colds or other minor illnesses,
such as coughing and fatigue.
This makes it difficult to diagnose until the disease has progressed to an advanced stage\cite{who_lung_cancer}.
\\

% genetical changes in cancer
\textbf{Gene expression} is a crucial factor in the investigation of cancer.
It describes the process by which genes are read and utilized within a cell to synthesize proteins
that regulate cellular growth and other essential processes.
Alterations in gene expression can contribute to the uncontrolled multiplication and abnormal behavior of cancer cells,
which ultimately leads to the development and progression of cancer.
\\

\textbf{Transcripts per Million (TPM)} is a measure of gene expression in a cell.
% It indicates the number of RNA copies of a particular gene per million total transcribed RNA molecules present in a sample.
A higher TPM value signifies that the corresponding gene is actively expressed and, consequently, produces more protein.
Through this method, it is possible to determine precisely which genes are activated within a cell and
their involvement in cancer development.

%TODO Multi Omics Data?!
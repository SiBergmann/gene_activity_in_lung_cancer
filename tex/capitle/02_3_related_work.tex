\subsection{Related work} \label{subsec:related_work}
The investigation of tumors and the identification of biomarkers for diagnosis and treatment are crucial tasks in oncology~\cite{das2024biomarkers}.
Over the past few years, various approaches have been explored to address these challenges.
Notably, graph-based methods have gained popularity for analyzing genetic data.
This section presents a brief insight of three studies in this field from the last years,
focusing on the use of graph databases and their algorithms for biological applications and
cancer research~\cite{rout2024systematic, simpson2020applying, shan2020network}.
By reviewing existing literature, we aim to provide a comprehensive understanding of the current state of the art in this area.
\\

% A systematic review of graph-based explorations of PPI networks: methods, resources, and best practices
The recent systematic review by Rout et al. (2024)~\cite{rout2024systematic} provides an extensive overview of
various graph-based methodologies for analyzing PPI networks,
emphasizing best practices and common challenges in integrating multi-omics data.
The authors highlight the importance of graph databases in storing and querying large-scale biological networks
and essential graph algorithms such as centrality measures and community detection that are relevant to my work using PageRank.
\\

% Applying Graph Database Technology for Analyzing Perturbed Co-Expression Networks in Cancer
A study by Simpson et al. (2020)~\cite{simpson2020applying} investigated the molecular mechanisms underlying various cancer types
through the analysis of gene expression data sourced from the TCGA database.
The authors employed co-expression networks, using Pearson correlation to establish relationships between genes based on their expression patterns.
However, our thesis will take a distinct approach by concentrating specifically on lung cancer
and utilizing PPI networks as base instead of co-expression networks.
In contrast to Simpson et al.'s comprehensive application of multiple graph algorithms, including PageRank,
Louvain community detection and Dijkstra's algorithm in each cancer type, we will focus on a more targeted approach.
Specifically, we will apply the Pagerank algorithm to identify key genes within our network.
\\

% Network-based prioritization of cancer genes by integrative ranks from multi-omics data
Shang and Liu (2020)~\cite{shan2020network} proposed a method for prioritizing cancer genes called iRank,
which integrates various biological levels, including gene and protein expression, as well as PPI,
to identify hepatocellular carcinoma.
In contrast, we will focus on building a PPI network to analyze the interactions between proteins and also add their corresponding genes.
Unlike Shang and Liu's approach, which concentrates on the TCGA dataset, our work will utilize the Cell Model Passport dataset.
Similar to their approach, we will employ the PageRank algorithm to identify important genes in our analysis.
\\

In summary, the studies mentioned above provide valuable insights into the application of graph-based methods for analyzing biological data.
Each with a slightly different focus, they demonstrate the versatility of graph databases and algorithms in cancer research.
\\

\subsection{Result Analysis} \label{subsec:result_analysis}
% TODO J - Bisher bleibst du da fast ausschließlich bei den Genen, lässt aber deinen Ansatz etwas außer Acht
% TODO J - Mehr Interpretation, bisher nur Beschreibung

%general Notes
As the final milestone of our study goal we need to fulfill the last objective  (see~\ref{obj:validation}) by analyzing and
validating the results of the PageRank algorithm.


% 1 - NDRG2
The gene \textbf{NDRG2} (N-Myc downstream-regulated gene 2) plays a critical role as a tumor suppressor,
regulating cell growth and preventing the spread of cancer cells to other parts of the body.
Its importance is particularly pronounced in tissues that are more likely for tumorigenesis,
highlighting its potential as a key regulator of cellular behavior~\cite{Lee2022NDRG2}

Recent studies have shed light on NDRG2's role in human lung cancer,
revealing that low expression levels of this gene are associated with aggressive tumor behavior and poor prognosis in patients.
Specifically, research has demonstrated that NDRG2 acts as a crucial suppressor of tumor cell metastasis and
functions as a negative regulator of tumor progression~\cite{Li2013NDRG2}.
Furthermore, another study has shown that downregulation of NDRG2, particularly in small-cell lung cancer,
inhibits tumor progression, underscoring the gene's importance in this context~\cite{Ma2024NDRG2}.

Our analysis highlights NDRG2 as a potential key player in the lung cancer landscape,
with a PageRank score of 25.635 and a significant downregulation in lung cancer cells ($\Delta_{TPM}$ = -0.317).
This finding is consistent with previous studies showing that NDRG2 acts as a tumor suppressor,
inhibiting metastasis and tumor progression~\cite{Li2013NDRG2,Ma2024NDRG2}.

Notably, it boasts the highest delta TPM value among the top 10 genes,
indicating a substantial decrease in expression levels in lung cancer cells.
% TODO - more Interpretation
\newline

% 2 - DMKN
The gene \textbf{DMKN}, also known as Dermokine, is a gene involved in skin cell function and regulation.
Specifically, it plays a crucial role in maintaining the integrity of the outer layer of the skin
by regulating various cellular processes~\cite{Naso2007Deromokine}.

While DMKN has not been directly implicated in lung cancer, its connections to other cancers offer valuable insights into its potential function.
Notably, research has demonstrated that DMKN-$\alpha$ is overexpressed in pancreatic cancer~\cite{Zhang2022DMKN}
and the $\beta$ isoform is downregulated in skin cancer and serves as a biomarker for early-stage colorectal cancer~\cite{Hasegawa2012Dermokine}.

It has a PageRank score of 19.900 and is downregulated in lung cancer cells,
with a $\Delta_{TPM}$ value of -0.203, indicating further investigation into this gene's role is warranted.
Our findings suggest that DMKN's involvement in lung cancer may be linked to its role in maintaining tissue integrity,
potentially serving as a tumor suppressor.
The gene's downregulation in lung cancer cells could compromise the skin's protective barrier function, facilitating tumor growth and spread.
\newline

% 3 - CAPN3
The gene \textbf{CAPN3}, also known as Calpain 3, is an integral component of the Calpain system,
a complex network of gene products and proteins that play critical roles in cellular processes.
Primarily expressed in skeletal muscle cells, CAPN3 plays a pivotal role in maintaining muscle function and integrity~\cite{Spinozzi2021Calpain}.

Although CAPN3 has been extensively studied in the context of muscle diseases,
its involvement in cancer, particularly lung cancer, remains poorly understood.
However, research has shown that the Calpain system, which includes CAPN3, is implicated in cancer progression and metastasis~\cite{Storr2011Calpain}.
Notably, studies have demonstrated that another member of the Calpain system, Calpain 2, is upregulated in lung cancer~\cite{Xu2019Calpain}.

Our analysis suggests that CAPN3 may be a gene of interest in lung cancer research,
with a PageRank score of 18.136 and significant downregulation ($\Delta_{TPM}$ = -0.310) in lung cancer cells,
indicating its potential importance in the disease.
This finding could indicate that CAPN3's loss of function contributes to the disruption of muscle integrity and structure,
potentially facilitating tumor growth and metastasis.
The gene's significant downregulation in lung cancer cells (second-highest decrease in expression levels among the top 10 genes)
suggests a potential link between muscle dysfunction and cancer progression.
As studies have already shown that the Calpain system is involved in cancer progression and metastasis~\cite{Storr2011Calpain},
and especially Calpain 2 is involved in lung cancer~\cite{Xu2019Calpain}, there needs to be further research on CAPN3 in lung cancer.
\newline

% 4 - SLC6A1
The \textbf{SLC6A1} gene encodes a protein that functions as a GABA carrier,
playing a crucial role in regulating neurotransmitter levels at synapses between neurons~\cite{Chen2020SLC6A1}.

Research has demonstrated that SLC6A1 promotes cancer growth in certain types of tumors,
including clear cell renal cell carcinoma, ovarian cancer, and prostate cancer~\cite{Chen2020SLC6A1}.
This suggests a potential role for SLC6A1 in tumor progression and metastasis.

Our analysis found that SLC6A1 is downregulated in lung cancer cells,
with a PageRank score of 14.170 and a $\Delta_{TPM}$ value of -0.267.
This result appears counterintuitive, given SLC6A1's established role as an oncogene in other types of cancer~\cite{Chen2020SLC6A1}.
Further investigation into the mechanisms underlying SLC6A1's expression in lung cancer is warranted to clarify its role in this disease.
\newline

%5 - SLC1A2
The \textbf{SLC1A2} gene encodes a protein that functions as a glutamate transporter,
playing a crucial role in regulating glutamate levels in the brain by removing glutamate from synapses~\cite{NCBI2017SLC1A2}.

Research has identified SLC1A2 as a partner gene in fusion genes associated with certain types of cancer.
Notably, studies have found that CD44-SLC1A2 fusions are present in gastric cancer~\cite{Tao2011CD44} and
primary colorectal cancer~\cite{Shinmura2015CD44}, indicating its potential role in cancer development.
Additionally, an analogous APIP/SLC1A2 fusion has been identified in colon cancer~\cite{Giacomini2013Breakpoint}.
Notably, these gene fusions were not detected in non-small cell lung cancers~\cite{Shinmura2015CD44}.

In contrast to SLC1A2's association with oncogenic fusion genes in other cancers,
our analysis reveals that it is downregulated in lung cancer cells,
with a PageRank score of 13.729 and a $\Delta_{TPM}$value of -0.233.
This finding raises questions about the gene's role in lung cancer and warrants further investigation.
The downregulation of SLC1A2 in lung cancer may indicate that the tumor suppressive effects of this gene are lost or disrupted,
contributing to tumor progression.
\newline

%6 - BRCA1
The \textbf{BRCA1} gene, also known as Breast Cancer Gene 1, is a tumor suppressor gene
that produces a protein involved in repairing damaged DNA\@.
This process is essential for maintaining genome stability by facilitating the correct repair of DNA breaks~\cite{NCI2020BRCA1}.

BRCA1 has been associated with various types of cancer, including breast and ovarian cancers~\cite{Lee2020BRCA1}.
However, its relationship to lung cancer is more complex.
Initial research suggested a potential link between overexpressed BRCA1 and
poor survival in non-small cell lung cancer patients~\cite{Rosell2007BRCA1}.
However, subsequent studies have contradicted this notion,
saying that BRCA1 does not play a role in NSCLC~\cite{Gachechiladze2012BRCA1,Lee2020BRCA1}.

Interestingly, our analysis reveals a different pattern for BRCA1 in lung cancer cells.
With a PageRank score of 13.288 and a $\Delta_{TPM}$ value of 0.214,
BRCA1 is upregulated in these cells, making it the only gene in our top 10 list to exhibit this behavior.
The upregulation of BRCA1 in lung cancer may indicate a distinct molecular mechanism at play,
potentially involving the activation of tumor-promoting pathways or the suppression of normal DNA repair functions.
Given BRCA1's role in maintaining genome stability,
its overexpression in lung cancer may indicate a compensatory response to genomic stress or other forms of cellular damage.
\newline

%7 - FHL1
The \textbf{FHL1} (four and a half LIM domains 1) gene encodes a protein
involved in regulating muscle cell differentiation and maturation,
playing a crucial role in maintaining proper muscle function.
It is primarily expressed in striated muscles but also found in other tissues such as the brain and testis~\cite{Storey2020FHL1}.

FHL1 has been implicated in various types of cancer, including breast, kidney, prostate and gastric cancer.
Research suggests that FHL1 expression is often reduced in these cancers,
which may contribute to their development~\cite{Li2008FHL1, Sakashita2008FHL1}.
Specifically, studies have found that FHL1 is downregulated in NSCLC patients~\cite{Niu2012FHL1}.

Our analysis reveals that FHL1 has a PageRank score of 12.849 and is downregulated in lung cancer cells,
with a $\Delta_{TPM}$ value of -0.217.
This finding is consistent with previous studies suggesting that FHL1 expression is reduced in various types of cancer,
including NSCLC, which may contribute to tumor development.
The loss of FHL1's tumor suppressive effects could potentially contribute to the aggressiveness and metastatic potential of NSCLC.
\newline

%8 - MBP
The \textbf{MBP} (Myelin Basic Protein) gene plays a crucial role in producing proteins essential for creating and
maintaining the protective myelin covering around nerve fibers, as well as
facilitating repair of damaged nerve coverings and regulating immune cell responses to infections or inflammation~\cite{Nye1995MBP}.

MBP has been linked to cancer research, particularly in the context of brain tumors.
Studies have suggested that MBP levels are elevated in patients with brain cancer and may serve as a potential biomarker for diagnosis.
However, increased MBP levels have also been observed in other brain diseases, such as multiple sclerosis~\cite{Zavialova2017MBP}.
Furthermore, research has found a correlation between MBP expression levels and brain metastasis from lung cancer~\cite{Nakagawa1994MBP}.

Interestingly, our analysis reveals that MBP is downregulated in lung cancer cells with a $\Delta_{TPM}$ value of -0.296.
This finding contradicts previous research suggesting elevated MBP levels in brain tumors,
including those metastasizing from lung cancer.
The discrepancy between these findings raises questions about the role of MBP in lung cancer and its relationship to brain metastasis.
One possible explanation is that the downregulation of MBP in lung cancer cells may be a result of epigenetic modifications
or other regulatory mechanisms that prevent the normal expression of this gene.
Further investigation is needed to fully understand the role of MBP in lung cancer biology and its potential as a biomarker for diagnosis.
\newline

%9 - ELN
The \textbf{ELN} gene encodes for the protein Elastin,
a key component of connective tissue that provides elasticity and
flexibility to tissues such as skin, lungs, and blood vessels~\cite{Debelle1999ELN}.

According to studies, elastin plays a significant role in cancer development.
Research has shown that elastin gene expression is increased in tumors from colorectal cancer patients~\cite{Li2020ELN}.
In breast cancer, elastosis, a condition characterized by an abnormal increase in elastin fibers,
is a common feature that increases with tumor progression~\cite{Lepucki2022ELN}.
Additionally, studies have found that Elastin in gastric cancer tissues is linked to changes that enable cancer cells
to spread more easily~\cite{Fang2023ELN}.

Our findings indicate that ELN expression is downregulated in lung cancer cells,
a paradoxical finding given its upregulation in other types of cancers.
With a $\Delta_{TPM}$ value of -0.246, this unexpected result could suggest a unique role for Elastin in the context of lung cancer.
The divergence from the established pattern of elastin's involvement in cancer development implies
that lung cancer may exploit distinct pathways to regulate elastin expression.
\newline

%10 - ARPP21
The \textbf{ARPP21} (cAMP regulated phosphoprotein 21) gene is involved in regulating cellular processes,
particularly in brain development and function.
It plays a crucial role in shaping the complexity of neurons during development~\cite{Rehfeld2018ARPP21}.

In addition to its direct impact on brain development, research has uncovered an indirect link between ARPP21 and cancer.
Specifically, the microRNA produced by ARPP21, miR-128, has been implicated in various types of cancer~\cite{Li2013ARPP21}.
Notably, studies have shown that miR-128 is downregulated in prostate cancer~\cite{Khan2010ARPP21}, ovarian cancer~\cite{Woo2012ARPP21},
and breast cancer~\cite{Zhu2011ARPP21}.

While the direct connection between ARPP21 and cancer remains unclear,
its connection to miR-128 suggests an indirect role for ARPP21 in cancer development through this microRNA.
Our analysis provides further insight into this relationship, revealing that ARPP21 is downregulated in lung cancer cells,
with a $\Delta_{TPM}$ value of -0.201.
This result stands in contrast to the upregulation of miR-128 observed in other types of cancer, such as breast and ovarian cancer.
Further investigation is needed to elucidate the mechanisms by which ARPP21 expression influences lung cancer biology,
and whether its downregulation in lung cancer cells has implications for tumor progression or response to treatment.
\newline


% Conclusion
In conclusion, our analysis has revealed a diverse set of top-ranked genes for lung cancer biomarkers,
each with unique characteristics and relationships to lung cancer.
The genes are involved in various cellular processes, including cell growth, muscle function, and neurotransmitter regulation.
While some genes, such as NDRG2 and MBP, exhibit direct connections to lung cancer,
others display complex or paradoxical roles in the disease.

Interestingly, even among the top-ranked genes, their connection to lung cancer is not always clear-cut.
For instance, BRCA1 has a controversial role in lung cancer, with some studies suggesting that it may act as an oncogene,
while others propose a tumor suppressor function.
Moreover, several genes on our list remain understudied in the context of lung cancer,
underscoring the necessity of continued research to elucidate their potential involvement.

Our findings collectively emphasize that lung cancer is a multifaceted disease with a complex genetic underpinning,
and underscore the importance of interdisciplinary approaches and continued investigation into the roles of these genes in disease progression.
By shedding light on these previously understudied genes, our analysis highlights new avenues for research and potential therapeutic targets in lung cancer.
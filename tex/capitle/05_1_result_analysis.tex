\subsection{Analysis} \label{subsec:analysis}
%general Notes
A first glance at the dataset of all genes reveals the complexity of gene expression in our study.
To establish a foundation for interpreting the results of the top 10 most relevant genes,
we examined the distribution of Pagerank scores and $\Delta_{TPM}$ values across the entire dataset

The PageRank scores display an initial peak at approximately 67, which stands out as a significant outlier.
This is followed by a sharp decline to around 40, with subsequent scores showing progressively smaller differences
as they approach the mid-30s range.
Notably, once the scores fall to this level, the gaps between them become considerably narrower.
%TODO This overall distribution suggests a pattern of exponential or logarithmic decay, indicating that the majority of the scores are heavily concentrated towards the higher end.

The data show a symmetric distribution of $\Delta_{TPM}$ values, with a pronounced peak around zero, suggesting a central tendency near the mean.
The data ranges approximately from -0.709 to 0.423, indicating moderate variability, and appears to follow a normal-like distribution.
\ref{fig:03_02_delta_tpm}
\newline


% 1 - NDRG2
\textbf{NDRG2} (N-Myc downstream-regulated gene 2) is a tumor suppressor gene that helps control cell growth and
prevents cancer cells from spreading to other parts of the body.
It is particularly important in regulating cells in tissues that are more likely to develop tumors~\cite{Lee2022NDRG2}.

Our analysis reveals that NDRG2 has a PageRank score of 25.635 and is downregulated in lung cancer cells,
with a $\Delta_{TPM}$ value of -0.317.
It has the highest delta TPM value among the top 10 genes, indicating a significant decrease in expression levels in lung cancer cells.

NDRG2 has been increasingly studied for its role in human lung cancer.
Research indicates that low expression of NDRG2 is associated  with more aggressive tumor behavior and
poorer prognosis in lung cancer patients.
For example, studies have demonstrated that NDRG2 plays a crucial role in suppressing tumor cell metastasis and
functions as a negative regulator of tumor progression~\cite{Li2013NDRG2}.
Another study showed that the gene, particularly in small-cell lung cancer,
plays a role in inhibiting tumor progression when it is downregulated~\cite{Ma2024NDRG2}.
\newline


% 2 - DMKN
\textbf{DMKN}, also known as Dermokine, is a gene involved in skin cell function and regulation.
It helps regulate cellular processes in the outer layer of the skin, where it plays a role in maintaining skin health~\cite{Naso2007Deromokine}.

Our analysis reveals that DMKN has a PageRank score of 19.900 and is downregulated in lung cancer cells,
with a $\Delta_{TPM}$ value of -0.203.

There are no known studies linking DMKN to lung cancer, but it is studied in the context of skin, pancreas and colorectal cancer.
For example, research has shown that DMKN-$\alpha$ is overexpressed in pancreatic cancer~\cite{Zhang2022DMKN}.
Another study found that DMKN-$\beta$ is downregulated for skin cancer and is a known biomarker for early-stage colorectal cancer~\cite{Hasegawa2012Dermokine}.
\newline


% 3 - CAPN3
The \textbf{CAPN3} (calpain 3) gene is part of the Calpain system, which includes multiple gene products and proteins.
It is primarily expressed in skeletal muscle cells, where it plays a crucial role in muscle function~\cite{Spinozzi2021Calpain}.

Our analysis reveals that CAPN3 has a PageRank score of 18.136 and is downregulated in lung cancer cells,
with a $\Delta_{TPM}$ value of -0.310.
It has the second-highest decrease in expression levels among the top 10 genes.

CAPN3 has been studied in the context of muscle diseases, but there is limited research on its role in cancer, lung cancer in particular.
However, studies have shown that the Calpain system, which includes CAPN3, is involved in cancer progression and metastasis~\cite{Storr2011Calpain}.
For example, research has shown that Calpain 2, another member of the Calpain system, is upregulated in lung cancer~\cite{Xu2019Calpain}.
\newline

% 4 - SLC6A1
\textbf{SLC6A1} is the shorting for the gene solute carrier family 6 member 1,
which encodes a protein that acts as a GABA carrier,
helping to regulate the levels of this neurotransmitter at synapses between neurons~\cite{Chen2020SLC6A1}.

In our analysis the gene has a PageRank score of 14.170 and is downregulated in lung cancer cells,
with a $\Delta_{TPM}$ value of -0.267.

Studies have shown that SLC6A1 plays a role in promoting cancer growth in certain types of tumors,
including clear cell renal cell carcinoma, ovarian cancer and prostate cancer~\cite{Chen2020SLC6A1}
\newline

%5 - SLC1A2
The \textbf{SLC1A2}, is another of the family of solute transporter proteins, which is primarily known as glutamat transporter.
It plays a crucial role in regulating glutamate levels in the brain by removing glutamate from synapses.\cite{NCBI2017SLC1A2}

In our analysis the gene has a PageRank score of 13.729 and is downregulated in lung cancer cells,
with a $\Delta_{TPM}$ value of -0.233.

Research has identified SLC1A2 as a partner gene in fusion genes associated with certain types of cancer.
Studies have found that CD44-SLC1A2 fusions are present in gastric cancer~\cite{Tao2011CD44} and
primary colorectal cancer~\cite{Shinmura2015CD44}, indicating its potential role in cancer development.
Furthermore, an analogous APIP/SLC1A2 fusion has been identified in colon cancer~\cite{Giacomini2013Breakpoint}.
Notably, these gene fusions were not detected in non-small cell lung cancers~\cite{Shinmura2015CD44}
\newline

%6 - !!!BRCA1
The \textbf{BRCA1}, Breast Cancer Gene 1, gene is a tumor suppressor gene that produces a protein involved in repairing damaged DNA\@.
It plays a crucial role in maintaining genome stability by facilitating the correct repair of DNA breaks~\cite{NCI2020BRCA1}.

Our analysis reveals that BRCA1 has a PageRank score of 13.288 and is upregulated in lung cancer cells,
with a $\Delta_{TPM}$ value of 0.214.
In our Top 10 list it is the only gene that is upregulated in lung cancer cells.

BRCA1 is a tumor suppressor gene that has been associated with various types of cancer,
including breast and ovarian cancers\cite{Lee2020BRCA1}.
Research has also suggested a potential link between overexpressed BRCA1 and lung cancer,
with some studies indicating an association with poor survival in non-small cell lung cancer patients\cite{Rosell2007BRCA1}
However, more recent findings have contradicted this notion,
saying that BRCA1 does not play a role in NSCLC~\cite{Gachechiladze2012BRCA1,Lee2020BRCA1}
\newline

%7 - FHL1
The \textbf{FHL1} (four and a half LIM domains 1) gene encodes a protein
that plays a crucial role in regulating muscle cell differentiation and maturation.
It is primarily expressed in striated muscles but also found in other tissues such as the brain and testis~\cite{Storey2020FHL1}.

Our analysis reveals that FHL1 has a PageRank score of 12.849 and is downregulated in lung cancer cells,
with a $\Delta_{TPM}$ value of -0.217.

FHL1 has been implicated in various types of cancer, including breast, kidney, prostate and gastric cancer.
Research suggests that FHL1 expression is often reduced in these cancers,
which may contribute to their development\cite{Li2008FHL1, Sakashita2008FHL1}.
Specifically, studies have found that FHL1 is downregulated in NSCLC patients~\cite{Niu2012FHL1}.
\newline

%8 - MBP
The \textbf{MBP} (Myelin Basic Protein) gene helps produce proteins that are essential for creating and
maintaining the protective covering around nerve fibers, called myelin.
It also plays a role in helping the body repair damaged nerve coverings and
regulates how immune cells respond to infections or inflammation~\cite{Nye1995MBP}.

Our analysis reveals that MBP has a PageRank score of 12.848 and is downregulated in lung cancer cells,
with a $\Delta_{TPM}$ value of -0.296.

MBP has been linked to cancer research, particularly in the context of brain tumors.
Studies suggest that MBP levels are elevated in patients with brain cancer and may serve as a potential biomarker for diagnosis.
However, it's worth noting that increased MBP levels have also been observed in other brain diseases, such as multiple sclerosis~\cite{Zavialova2017MBP}.
Furthermore, there is evidence suggesting a correlation between MBP expression levels and brain metastasis from lung cancer~\cite{Nakagawa1994MBP}.
\newline

%9 - ELN
The \textbf{ELN} (Elastin) gene encodes for the protein Elastin, which is a key component of connective tissue that provides elasticity and
flexibility to tissues such as skin, lungs, and blood vessels~\cite{Debelle1999ELN}.

Our analysis reveals that ELN has a PageRank score of 11.119 and is downregulated in lung cancer cells,
with a $\Delta_{TPM}$ value of -0.246.

According to studies, elastin plays a significant role in cancer development.
For example, research has shown that elastin gene expression is increased in tumors from colorectal cancer patients~\cite{Li2020ELN}.
In breast cancer, elastosis, a condition characterized by an abnormal increase in elastin fibers,
is a common feature that increases with tumor progression~\cite{Lepucki2022ELN}.
Additionally, studies have found that Elastin in gastric cancer tissues is linked to changes that enable cancer cells
to spread more easily~\cite{Fang2023ELN}.
\newline

%10 - ARPP21
The \textbf{ARPP21} (cAMP regulated phosphoprotein 21) gene is involved in regulating cellular processes, particularly in brain development and function.
It plays a crucial role in shaping the complexity of neurons during development~\cite{Rehfeld2018ARPP21}.

Our analysis reveals that ARPP21 has a PageRank score of 11.089 and is downregulated in lung cancer cells,
with a $\Delta_{TPM}$ value of -0.201.

The gene ARPP21 is involved in producing a microRNA called miR-128, which has been implicated in various types of cancer~\cite{Li2013ARPP21}.
Studies have shown that miR-128 is downregulated in prostate cancer~\cite{Khan2010ARPP21}, ovarian cancer~\cite{Woo2012ARPP21},
and breast cancer~\cite{Zhu2011ARPP21}.
Although the direct link between ARPP21 and cancer remains unclear,
its connection to miR-128 suggests an indirect role for ARPP21 in cancer development.
\newline
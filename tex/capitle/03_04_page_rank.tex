\subsection{Graph Database Algorithm} \label{subsec:graph_database_algo}
As our final step for objective~\ref{obj:graph_algorithm},
we employ the \textbf{PageRank} algorithm to measure the importance of nodes within our graph database.
This algorithm is particularly well-suited for identifying genes that play a crucial role in the network,
enabling us to identify not only highly connected genes but also those with significant functional relevance.

For efficient computation and scalability,
we start with creating a projection of the entire graph prior to applying PageRank analysis to our large-scale graph database.

To perform the PageRank analysis, we use the following cypher query:
\begin{lstlisting}[language=Cypher, label={lst:pagerank}]
    CALL gds.pageRank.stream('gene_protein_graph')
    YIELD nodeId, score
    RETURN gds.util.asNode(nodeId).id AS Gene_ID,
           gds.util.asNode(nodeId).gene_name AS Gene_Name,
           score,
           gds.util.asNode(nodeId).Δ_TPM AS Δ_TPM,
           gds.util.asNode(nodeId).Δ_TPM_relevant AS Δ_TPM_relevant
    ORDER BY score DESC
\end{lstlisting}

This query returns a list of nodes, including genes and proteins, along with their respective PageRank scores.
% TODO - K - Here or in Result?
By filtering out the proteins from this list, we isolated the genes and
further refined them to include only those exhibiting a significant change in gene activity,
indicated by the value for $delta\_tpm\_relevant$.

The resulting subset of genes represents a group that has not only high connectivity
within the network but also exhibit a substantial difference in gene expression.
\subsection{Computational background} \label{subsec:computational_background}
% Graph databases
\textbf{Graph databases} are an effective way to model and analyze graph-like data,
which is in contrast to traditional relational databases well-suited for applications involving relationships between entities.
For instance, social networks and biological systems can be effectively represented using these graph structures~\cite{graph_db_survey}.
In particular, graph databases have been widely used in biological data analysis, such as modeling Protein-Protein-Interaction (PPI) networks
that illustrate the interaction between proteins to control cellular processes.
In this thesis, we use a graph database to put our collected cancer data into a meaningful context.

Just like a graph, a graph database consists of two basic components: nodes and edges between nodes.
These components form the foundation for storing and analyzing large amounts of data.

Nodes are the units that store the basic information of a certain type of entity~\cite{graph_db_survey}.
In our case, genes and proteins are represented as two different node types in the database.
Each gene is, therefore, a node with properties, such as an $ID$, a $name$ or a $TPM$ value for its activity.

Relationships between two nodes are represented by an edge, which also can be of different types.
In our graph database, for example, there are $interactions$ as edges between proteins and $connections$ as edges between a gene and a protein.
As in common graph theory, edges can represent structures like one-to-many or many-to-many relationships.
Edges can also have weights, directions, or properties like $name$ or $type$~\cite{graph_db_power_limitations}.
We do not use these additional metadata in our setup, but they could be used to model more complex relationships.

We will implement our graph database using Neo4J, since it is a widely used graph database management system.
\\

\textbf{Graph database algorithms} enable the efficient analysis and extraction of meaningful insights from large-scale network data.
They can be broadly categorized into three primary groups, which provide different perspectives on graph structure.

Traversal and Pathfinding Algorithms enable the identification of shortest or
most optimal paths between nodes in the graph, thereby facilitating the exploration of network topology.
Examples include Depth-First-Search, Breadth-First-Search, Shortest Path, and Max-Flow-Min-Cut~\cite{neo4j_graph_algorithms}.

Centrality Algorithms are instrumental in understanding which nodes within a graph hold significant importance.
By evaluating centrality measures such as degree centrality, closeness centrality, betweenness centrality,
and PageRank, valuable insights into the underlying network structure is gained~\cite{neo4j_graph_algorithms}.

Community Detection Algorithms, also known as clustering or partitioning algorithms,
are essential for identifying groups of nodes within a graph that share similar characteristics
or exhibit extensive connectivity.
Examples include Label Propagation, Louvain Modularity, and Strongly Connected Components~\cite{neo4j_graph_algorithms}.
\\


% PageRank algorithm
In our analysis, we utilize the \textbf{PageRank algorithm} to identify influential nodes within the network
based on their connectivity and centrality.
The PageRank algorithm was originally a method for evaluating and prioritizing websites on the internet.
It was developed by Larry Page and Sergey Brin in 1988 at Stanford University~\cite{page1999pagerank},
the founders of Google.
The general idea behind the algorithm was that the more links pointing to a website, the more important it is.
This concept is extended in the PageRank algorithm, which assigns a node's score by iteratively considering not only its direct connections
but also the indirect relationships through its linked neighbors.

The algorithm calculates a PageRank score per node using the following formula:

\begin{align*}
PR(A)=(1-d)+d(\frac{PR(T_1)}{C(T_1)}+\dots+\frac{PR(T_n)}{C(T_n)})
\end{align*}

where $PR(A)$ is the PageRank score of node $A$,
$T_1$ to $T_n$ are nodes with edges to $A$,
$d$ is the damping factor,
and $C(T_1)$ represents the number of edges to node $T_1$~\cite{neo4j_graph_algorithms}.

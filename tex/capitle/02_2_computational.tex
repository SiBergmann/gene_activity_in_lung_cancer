\subsection{Computational background} \label{subsec:computational_background}
% Graph databases
\textbf{Graph databases} are an effective way to model and analyze complex data structures.
In the thesis, we use graph databases to put the collected cancer data into a meaningful context.
A graph database consists of two basic components: Nodes and edges.
Nodes are the units that store the basic information of a certain type of object.
In our case, it is genes and proteins that are represented as nodes in the database.
Each gene is therefore a node with its own properties,
such as an ID, a name or a TPM value for its activity.
Relationships between the nodes are represented by edges, which can represent different types of connections.
In our graph database, for example, there are “interactions” as connections between proteins.
Edges can represent both one-to-many and many-to-many relationships,
which allows us to model complex relationships between nodes.
Edges also can have weights or properties or a direction.
We do not use these properties in our database,
but they could be used to model more complex relationships between nodes.
Graph databases provide several benefits when working with complex networked data.
Notably, they enable efficient querying and visualization of complex relationships,
facilitating a deeper understanding of these networks.

In biological data analysis, graph databases are often used to analyze protein-protein interaction (PPI) networks.
These networks model the interaction between proteins to control cellular processes.
By using a graph database, we can visualize and query these networks more efficiently,
allowing us to gain important insights into cellular processes.
[PAPER XY]

% TODO zusammenlegen
In our project, we take advantage of graph databases to analyze the graph-like connections between genes and proteins.
This allows us to represent and query the complex relationships more efficiently.
Part of our work also involves creating and using a PPI\@.
For our project, we used Neo4J, one of the best-known and most powerful graph databases.
It offers a high degree of flexibility and enables us to manage our data efficiently.
\\

% TODO better intro
\textbf{Graph database algorithms} can be broadly categorized into three primary groups.
These categories facilitate the efficient analysis of complex networks
by providing different perspectives on graph structure.

Traversal and Pathfinding Algorithms enable the identification of shortest or
most optimal paths between nodes in the graph, thereby facilitating the exploration of network topology.
Examples include Depth-First-Search, Breadth-First-Search, Shortest Path, and Max-Flow-Min-Cut.

Centrality Algorithms are instrumental in understanding which nodes within a graph hold significant importance.
By evaluating centrality measures such as degree centrality, closeness centrality, betweenness centrality,
and PageRank, valuable insights into the underlying network structure is gained.

Community Detection Algorithms, also known as clustering or partitioning algorithms,
are essential for identifying groups of nodes within a graph that share similar characteristics
or exhibit extensive connectivity.
Examples include Label Propagation, Louvain Modularity, and Strongly Connected Components.
\cite{neo4j_graph_algorithms}\\


% PageRank algorithm
In our analysis of gene activity networks,
we utilize the \textbf{PageRank algorithm} to identify influential genes based on their connectivity and
centrality within the graph.
The PageRank score per gene is calculated using the following formula:

\begin{align*}
PR(A)=(1-d)+d(\frac{PR(T_1)}{C(T_1)}+...+\frac{PR(T_n)}{C(T_n)})
\end{align*}

where $PR(A)$ is the PageRank score of node $A$,
$T_1$ to $T_n$ are nodes with edges to $A$,
$d$ is the damping factor,
and $C(T_1)$ represents the number of edges to node $T_1$.

The PageRank algorithm was originally a method for evaluating and prioritizing websites on the internet.
It was developed in 1988 at Stanford University by Larry Page and Sergey Brin~\cite{page1999pagerank},
the founders of Google.
The idea behind the algorithm is that the more links pointing to a website, the more important it is.

The PageRank algorithm assigns a higher score to genes that are connected to many other important genes,
indicating their central role in the network.
This allows us to identify key regulatory genes.
By leveraging Neo4J's graph database capabilities,
we can efficiently compute PageRank scores for our gene activity network,
providing valuable insights into the complex relationships between genes.

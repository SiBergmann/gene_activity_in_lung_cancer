\begin{abstract} \label{abstract}

    Lung cancer remains one of the leading causes of mortality worldwide,
    with significant implications for public health and individual well-being.
    To address this challenge,
    we aim to contribute to the ongoing efforts to combat lung cancer by identifying potential biomarkers using graph databases and algorithms.
    Traditional methods often struggle to capture the complex relationships between genes and proteins.
    We try to overcome these challenges with a graph based approach.
    By analyzing gene expression data and applying a network-based approach,
    we have successfully identified 10 genes that may serve as biomarkers for early detection or personalized treatment.
    Our method integrates over 17,000 genes, 100,000 proteins, and 13,000,000 interactions,
    allowing us to pinpoint relevant genes with high connectivity in the network.
    This approach has revealed promising new targets for further research and
    underscore the importance of considering gene interactions in cancer development.
    While our study provides valuable insights into the genetic landscape of lung cancer,
    it is essential to note that further validation and refinement of our methodology are necessary to ensure the accuracy and reliability of our findings.

\end{abstract}
\section{Conclusion} \label{sec:conclusion}
% key findings:
Our study aimed to identify potential biomarkers for lung cancer by analyzing the gene network associated with its development
using the PageRank algorithm.
We focused on the top 10 genes with significant change in gene activity,
which were ranked based on their connectivity within the network.

The results of our analysis revealed that some of these genes have been previously associated with cancer development,
such as NDRG2 and MBP, while others are less well-known and may represent novel targets for further research.
Our study highlights the potential of a network-based approach in identifying biomarkers for lung cancer and
underscores the importance of considering gene interactions in the context of cancer development.


% limitations:
However, our analysis also revealed some limitations in our approach,
particularly in the classification of genes as $\Delta TPM relevant$ based on the $\Delta_{TPM}$ measure.
As we looked at some genes known for lung cancer, only one was classified as having relevant changes in gene activity.
This limitation highlights potential biases in our method and
underscores the need for further refinement of our approach to improve its sensitivity and specificity.


Same may be said about the PageRank algorithm, which while robust in identifying well-connected nodes within the network,
may not always capture the full complexity of gene interactions in cancer development.


% future work:
Future research directions include experimental studies to validate the role of novel target genes in lung cancer development,
as well as refining our methodology by incorporating additional measures or
using different datasets to validate our findings.
By continuing to explore the complex interactions between genes in lung cancer development,
we may be able to improve the accuracy and reliability of predictive models in lung cancer diagnostics and treatment.

% final conclusion
In conclusion, our study has successfully identified potential biomarkers for lung cancer using a network-based approach.
While our results provide insights into the genetic landscape of lung cancer,
further research is needed to validate these results and refine our methodology for identifying biomarkers.
By continuing to explore the complex interactions between genes in lung cancer development,
we may be able to improve the accuracy and reliability of predictive models in lung cancer diagnostics and treatment.
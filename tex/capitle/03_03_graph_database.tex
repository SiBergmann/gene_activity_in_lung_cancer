\subsection{Graph Database} \label{subsec:graph_database}
% Checked - all edges are created with 'match' statement
% todo check numbers of Protein Edges and Protein Nodes!! current number is only proteins directly on genes
% TODO Bezeichnung der IDs in Cypher Queries nochmal prüfen
% TODO check final version of the Cypher query for gene nodes with all attributes
% TODO [BILD] von PPI network und extended network mit Genen

% Intro
As we have created our data models for the graph database, our next objective is to create a graph database\ref{obj:graph_algorithm}
that can be used to perform the PageRank algorithm as a graph algorithm.
With four huge datasets at our disposal, optimizing the generation of the database is crucial.
In this section, we describe the queries, called cypher queries, used to create and populate our graph database,
including information about creating indexes, constraints, relationships between nodes, and loading data into the database.\\


% PPI Network
First, we create a \textbf{basic PPI network} by generating proteins and their interactions as edges.
To start with, we need to create protein nodes by creating an index on the protein property id for faster query execution.
Next we load the data as a list, create nodes in batches for efficient processing,
and finally populate the graph database with 101,731 protein nodes.

We don't require any additional properties for the protein nodes since they are solely used as connections between genes.
The Cypher query for creating protein nodes is:

\begin{lstlisting}[language=Cypher, label={lst:protein_nodes}]
    CREATE (p:protein {id: 'Protein ID'})
\end{lstlisting}

Next, we create interaction edges between these protein nodes by loading the data as a list of protein tuples,
searching for both protein nodes by their IDs, and creating an edge in batches for efficient processing.
This results in the creation of 11,247,242 edges between protein nodes.
The Cypher query for creating protein-protein edges is:

\begin{lstlisting}[language=Cypher, label={lst:protein_edges}]
    MATCH (s:protein{id:'left Protein ID'})
    MATCH (s:protein{id:'right Protein ID'})
    CREATE (s)-[:interaction]->(t)
\end{lstlisting}

At this point, we have a PPI network in place. \\


% Extending the Network with Genes
Our actual focus is on the genes.
For constructing our \textbf{extended PPI network} with genes on the edges of the protein nodes, we need to connect them to the proteins.
To achieve this, we first create gene nodes with their associated attributes,
utilizing efficient data loading techniques to facilitate faster query execution.
Specifically, we implement an index on the gene property id to expedite querying,
load the data as a list, create nodes in batches to optimize processing,
and ultimately populate the graph database with 17,626 gene nodes.

The Cypher query employed for creating these gene nodes is:
\begin{lstlisting}[language=Cypher, label={lst:gene_nodes}]
    CREATE (p:gene {
            id: 'id',
            gene_name: 'gene_name',
            norm_healthy_tpm: 'norm_healthy_tpm',
            norm_cancerous_tpm: 'norm_cancerous_tpm',
            delta_tpm: 'delta_tpm',
            delta_type: 'delta_type',
            delta_tpm_relevant: 'delta_tpm_relevant'})
\end{lstlisting}

The genes in our network will be characterized by several properties:
$gene\_name$, $norm\_healthy\_tpm$, $norm\_cancerous\_tpm$, $delta\_tpm$, and $delta\_type$.


Among these, the calculation of $delta\_tpm\_relevant$ stands out as a pivotal factor in our analysis.
Although other attributes may be less relevant to our current investigation, they retain potential value for future tasks.

To model relationships between genes and proteins,
we establish connections between these nodes by loading gene-protein interaction data as a list of tuples.
This involves matching both gene and protein nodes based on their respective Ids,
and creating edges in batches to optimize processing efficiency.
The result is a comprehensive network of 101,731 connections between gene and protein nodes

The Cypher query for creating gene-protein connection edges is:
\begin{lstlisting}[language=Cypher, label={lst:gene_protein_edges}]
    MATCH (s:protein{id:'Protein ID'})
    MATCH (s:gene{id:'Gene ID'})
    CREATE (s)-[:connection]-(t)
\end{lstlisting}

Through the execution of these Cypher queries,
we are able to populate our graph database with the necessary nodes and edges to perform the PageRank algorithm.\\
\subsection{Graph Database} \label{subsec:graph_database}
% Checked - all edges are created with 'match' statement
% todo check numbers of Protein Edges and Protein Nodes!! current number is only proteins directly on genes
% TODO Bezeichnung der IDs in Cypher Queries nochmal prüfen
% TODO check final version of the Cypher query for gene nodes with all attributes
% TODO [BILD] von PPI network und extended network mit Genen

% Intro
As we have created our data models for the graph database, our next objective is to create a graph database
that can be used to perform the PageRank algorithm as a graph algorithm.
With four huge datasets at our disposal, optimizing the generation of the database is crucial.

In this section, we describe the Cypher queries used to create and populate our graph database,
including information about creating indexes, constraints, relationships between nodes, and loading data into the database.\\


% PPI Network
First, we create a basic PPI network by generating proteins and their interactions as edges.
To start with, we need to create protein nodes with their respective properties.
This process involves creating an index on the protein property id for faster query execution, loading the data as a list,
creating nodes in batches for efficient processing, and finally populating the graph database with 101,731 protein nodes.

We don't require any additional properties for these protein nodes since they are solely used as connections between genes.
The Cypher query for creating protein nodes is:

\begin{lstlisting}[language=Cypher, label={lst:protein_nodes}]
    CREATE (p:protein {id: 'Protein ID'})
\end{lstlisting}

Next, we create interaction  edges between proteins by loading the data as a list of protein tuples,
searching for both protein nodes by their IDs, and creating an edge in batches for efficient processing.
This results in the creation of 11,247,242 edges between protein nodes.
The Cypher query for creating protein-protein edges is:

\begin{lstlisting}[language=Cypher, label={lst:protein_edges}]
    MATCH (s:protein{id:'left Protein ID'})
    MATCH (s:protein{id:'right Protein ID'})
    CREATE (s)-[:INTERACTS]->(t)
\end{lstlisting}

At this point, we have a PPI network in place. \\


% Extending the Network with Genes
Our actual focus is on the genes.
To get an extended network with genes on the edges of the protein nodes, we need to link them to the proteins.
We start by creating gene nodes with their respective properties.
This process involves creating an index on the gene property id for faster query execution,
loading the data as a list, creating nodes in batches for efficient processing, and finally populating the graph database with 17,626 gene nodes.

The Cypher query for creating gene nodes is:
\begin{lstlisting}[language=Cypher, label={lst:gene_nodes}]
    CREATE (p:gene {
            id: 'id',
            gene_name: 'gene_name',
            norm_healthy_tpm: 'norm_healthy_tpm',
            norm_cancerous_tpm: 'norm_cancerous_tpm',
            delta_tpm: 'delta_tpm',
            delta_type: 'delta_type',
            delta_tpm_relevant: 'delta_tpm_relevant'})
\end{lstlisting}

The genes have several properties, including $gene\_name$, $norm\_healthy\_tpm$, $norm\_cancerous\_tpm$, $delta\_tpm$, and $delta\_type$.
These attributes are used to calculate a more relevant attribute called $delta\_tpm\_relevant$, which will be the primary focus of our PageRank analysis.
While the other attributes may not be directly relevant to our current analysis, they could potentially be useful for future tasks.

To represent relationships between genes and proteins, we create edges called connections by loading the data as a list of gene-protein tuples
and matching both gene and protein nodes by their Ids.
This process involves creating edges in batches for efficient processing, resulting in 101,731 connections between gene and protein nodes.

The Cypher query for creating gene-protein connection edges is:
\begin{lstlisting}[language=Cypher, label={lst:gene_protein_edges}]
    MATCH (s:protein{id:'Protein ID'})
    MATCH (s:gene{id:'Gene ID'})
    CREATE (s)-[:CONNECTION]-(t)
\end{lstlisting}

By executing these Cypher queries, we can populate our graph database with the necessary nodes and edges to perform the PageRank algorithm.\\



% PAGERANK
% TODO Analysis of the difference between histogramm of all and histofgramm of relevant genes

As our final step for the second objective, we employed the \textbf{PageRank} algorithm to measure the importance of nodes in our graph database.
This algorithm is particularly useful for identifying key genes that play a crucial role in the network
because it can accurately capture the hierarchical structure of gene interactions,
allowing us to identify not only highly connected genes but also those with significant functional relevance.
For applying PageRank analysis to our large-scale graph database,
we create a projection of the entire graph, which enables efficient computation and scalability.

To perform the PageRank analysis, we used the following Cypher query:

\begin{lstlisting}[language=Cypher, label={lst:pagerank}]
    CALL gds.pageRank.stream('gene_protein_graph')
    YIELD nodeId, score
    RETURN gds.util.asNode(nodeId).id AS node,
           gds.util.asNode(nodeId).gene_name AS gene_name,
           score,
           gds.util.asNode(nodeId).delta_tpm AS delta_tpm,
           gds.util.asNode(nodeId).delta_tpm_relevant AS delta_tpm_relevant
    ORDER BY score DESC
\end{lstlisting}

This query yielded a list of nodes, including genes and proteins, along with their respective PageRank scores.
By dropping the proteins from this list, we focused on the genes and
further filtered them by selecting only those with a significant change in gene activity,
indicated by the value for $delta\_tpm\_relevant$.

The resulting list of genes represents those that have not only high connectivity
within the network but also exhibit a substantial difference in gene expression.
This subset was visualized using histograms, which demonstrate the distribution of PageRank scores for all genes
(Figure~\ref{fig:03_02_hist_pagerank}) and only for relevant genes (Figure~\ref{fig:03_02_hist_pagerank_relevant}).

\begin{figure}[h]
\minipage{0.45\textwidth}
    \includegraphics[width=\linewidth]{figures/03_03_hist_pagerank}
    \caption{Histogram of Pagerank Scores for all genes}
	\label{fig:03_02_hist_pagerank}
\endminipage
\hfill
\minipage{0.45\textwidth}
  \includegraphics[width=\linewidth]{figures/03_03_hist_pagerank_relevant}
  \caption{Histogram of Pagerank Scores only for relevant genes}
	\label{fig:03_02_hist_pagerank_relevant}
\endminipage
\end{figure}

Lastly we filtered the top 10 genes with the highest PageRank scores and a significant change in gene activity.


% Conclusion
With this process, we have successfully done our second objective~\ref{obj:graph_algorithm},
which was to apply a graph algorithm to identify key genes in the network.



\subsection{Data} \label{subsec:data}
% This subsection should focus on describing the original datasets used in your study, their characteristics,
% and any preprocessing steps taken to prepare them for analysis.
% It includes information about the GTEx and CMP datasets, how they were downloaded, formatted, and processed.

% Description of the original datasets used (GTEx and CMP)
% Characteristics of the datasets (number of genes, tissues, TPM values, etc.)
% Steps taken to download and preprocess the data
% Data quality control and any issues encountered
% Methods used to handle missing or duplicate data
% Any transformations or normalizations applied to the data

% TODO wieso habe ich diese Datasets gewählt? Ehr Abgrenzung?
% TODO Warum habe ich TPM wert genommen?

Our study consists of two main datasets.
One includes healthy tissue data from the Genotype-Tissue Expression (GTEx) project,
and the other includes cancerous tissue data from the Cell Model Passport (CMP) project.
These two datasets are the base for the gene nodes in our graph database.
For these, we need to have a single table with each gene as row and a single value for healthy and cancerous tpm values for every gene in every tissue.
To have a unique name for every gene, we use the Ensemble ID (ENS ID) as the primary key for the gene nodes.
These are a unique erkennung for genes, proteins and other genetic elements collected in the Ensemble database
from 1999 by the European Bioinformatics Institute and the Wellcome Trust Sanger Institute \cite{ensembl_project}.
\\

% Genotype-Tissue Expression (GTEx) dataset
% todo file name too long for line
For the healthy tissue samples used in this study,
we downloaded the \texttt{GTEx\_Analysis\_2017-06-05\_v8\_RNASeQCv1.1.9\_gene\_tpm} dataset from the GTEx portal \cite{gtex_download}.
The \textbf{GTEx portal} is a large-scale, publicly available resource for studying gene activity.
The Adult GTEx project aims to characterize the gene expression patterns in healthy tissues across different individuals and ages,
providing valuable insights into the underlying biology of human development and disease.
This dataset is part of the Bulk Tissue Expression data in the V8 release,
which provides RNA sequencing data for a large number of tissue samples.
% genommen weil in der Wissenschaft üblich


The dataset is a .gct file that contains TPM values for 56,156 genes identified by ENS ID as rows in 17,382 different tissues as columns.
The data was initially stored in a file in wide format where each tissue had its own column.
% TODO more?
The TPM values for the tissues are between 0 and 747,400.
Since there are no missing values in the dataset, we did not need to handle any missing data.
To process this data into a suitable format, we employed the following steps:
\begin{itemize}
    \item \textbf{Reshaping the data to length format:} We read the data from the original file in chunks of 3,000 rows at a time to avoid memory issues.
    For each chunk of data, we transformed the columns for each tissue into individual rows,
    resulting in a dataset with three columns (ENS ID, Tissue, TPM) and 56,156 * 17,382 rows.
    \item \textbf{Grouping by genes:} Once all chunks had been processed, we separated the combined dataset in new chunks of approximately 200 million rows.
    These chunks have been grouped by gene using an aggregate function that calculated both the sum and count of TPM values for each gene.
    \item \textbf{Calculating mean tpm:} To handle genes that had been split across multiple chunks,
    we performed a global aggregation on the genes of the sum of the dataset.
    We then calculated the mean TPM value for each gene by dividing the sum by the count of each observation.
    %TODO unklar
\end{itemize}
The resulting dataset with 56,156 genes and a mean TPM value for every gene was saved as a CSV file for further processing.
% TODO insert example Rows - as image or as text - Schaubild mit Steps
\\


% Cell Model Passport
The cancerous tissue samples used in this study were obtained from the \textbf{Cell Model Passport (CMP) project}.
The CMP project...

The used file was downloaded from the CMP portal \cite{cmp_download} under the section Expression Data
with the name \texttt{rnaseq\_all\_data\_20220624}.
It contains data from the Sanger Institute and as well as from the Broad Institute.
% TODO Mehr Infos ? https://depmap.sanger.ac.uk/documentation/datasets/expression/
The file is a .csv file that contains TPM values for genes infected with different cancer types.
The data was initially stored in long format where CMP ID for genes, Tissues, TPM Values and more information.



{\color{lightgray}
    * CMP (Cell Model Passport) - cancerous tissues
        * downloaded all RNA-Seq processed data - which contain data from Sanger Institute and from Broad Institute
        * format was: genes with CMP IDs as rows and tpms, tissues (called model), ... per column
        * separate file loaded with model annotation to get a list of all models (tissues) that have lung cancer as cancer type
        * filtered the data to get only the lung cancer models
        * grouped all models (tissues) to get a mean tpm value for every gene
        * needed to add ENS ID to the genes for further processing
        * separate ensemble file downloaded from biomart (https://www.ensembl.org/biomart/martview/)
        * contains Gene stable ID and gene symbol (name)
        * idea is to add the Gene stable ID by merging both tables by the gene name
        * daher gehen wir davon aus, dass jeder gen name eindeutig ist
        * war es nicht - 10.605/48.311 rows in the ensemble file had a not unique gene name
        * rows with names that are not unique were removed
        * merging the ensemble file with the CMP data to get the ENS ID for every gene
        * 3.760 / 37.262 genes had no ENS ID → missing IDs could be added by synonymes maybe because of duplicate removement
        * these genes are removed
    → 33.502 human genes with mean values tpm for lung cancer
}
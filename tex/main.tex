\documentclass[11pt,a4paper]{article}

\usepackage[margin=2cm]{geometry}
\usepackage[utf8]{inputenc}
\usepackage[english]{babel}
\usepackage{amsmath}
\usepackage{amsfonts}
\usepackage{amssymb}
\usepackage{graphicx}
\usepackage{hyperref}

%added userpackages
\usepackage{microtype}
\usepackage{natbib}
\usepackage{lmodern}
\usepackage{bmpsize-base}



%%%%%%%%%%%%%%%%%%%%%%%%%

\title{Graph-Based Analysis of Gene Activity in Lung Cancer based on Protein-Interaction Networks}
\author{
Simone Bergmann, 3957624 \\
\texttt{simone.bergmann@uni-bielefeld.de}%
}
%%%%%%%%%%%%%%%%%%%%%%%%%
\twocolumn
\begin{document}
\maketitle

% FACTS:
% 15 bis 30 Seiten (exklusive Titel, Abstract, Inhaltsverzeichnis, Literaturverzeichnis, ...)
% Deckblatt
% Inhaltsverzeichnis
% Nicht zu viele Unterkapitel / zu sehr verschachtelt
% ca 30 Quellen
% Jedes Kapitel am Anfang und am Ende zusammenfassen aber Fokus auf Ende

\cite{page1999pagerank}

Task:
Creation of a data model and integration of genes and proteins with interactions in an appropriate form within a graph database,
which is to be used for the identification of relevant genes for distinguishing between healthy tissue and lung cancer.

Key facts:
* cancerous mean tpm from Cell Model Passport
* healthy mean tpm from Genotype-Tissue Expression
* delta tpm as difference between healthy and cancerous tpm
* genes are connected with proteins and proteins are connected with each other (STRING database)
* neo4J as graph database
* pagerank as graph algorithm to identify relevant genes

% TODO
% Abstact soll kursiv sein

\begin{abstract} \label{abstract}

    Lung cancer remains one of the leading causes of mortality worldwide,
    with significant implications for public health and individual well-being.
    To address this challenge,
    we aim to contribute to the ongoing efforts to combat lung cancer by identifying potential biomarkers using graph databases and algorithms.
    Traditional methods often struggle to capture the complex relationships between genes and proteins.
    We try to overcome these challenges with a graph based approach.
    By analyzing gene expression data and applying a network-based approach,
    we have successfully identified 10 genes that may serve as biomarkers for early detection or personalized treatment.
    Our method integrates over 17,000 genes, 100,000 proteins, and 13,000,000 interactions,
    allowing us to pinpoint relevant genes with high connectivity in the network.
    This approach has revealed promising new targets for further research and
    underscore the importance of considering gene interactions in cancer development.
    While our study provides valuable insights into the genetic landscape of lung cancer,
    it is essential to note that further validation and refinement of our methodology are necessary to ensure the accuracy and reliability of our findings.

\end{abstract}

\section{Introduction} \label{sec:intro}
% TODO Schicker formulieren weil erster Satz der Arbeti oder weglassen
% In the Introduction, the motivation for the work is presented, followed by the goals of the thesis.
%The structure of the thesis is outlined and the limitations of the work are discussed.

% TODO könnte noch etwas länger sein
\subsection{Motivation} \label{subsec:motivation}
% 300 Words

% Problem
Lung cancer is a major public health problem worldwide, caused by various factors,
including smoking, air pollution, and also genetic factors.
% TODO J - Quelle
It is the leading cause of cancer-related deaths worldwide\cite{ferlay2024global}.
Despite advances in medical care, the survival rate for patients remains relatively low,
with only $30.2\%$ of women and $22.1\%$ of men (2016)
surviving beyond five years after diagnosis~\cite{seer2024explorer}, often due to late-stage detection.
Therfore, early detection is crucial to initiate timely treatment and ultimately reduce the mortality rate.
\\
% TODO J - Übergang

% Solution
We focus on analyzing gene expression data to identify potential biomarkers for lung cancer.
We use graph databases for this purpose,
as they offer ways to represent complex relationships between large sets of genes and proteins
and analyze them efficiently using graph based algorithms.
These data provide valuable insights into the underlying molecular mechanisms of lung cancer
and help to improve the early diagnosis and treatment of lung cancer.

\subsection{Problem Statement and Goals} \label{subsec:problem_statement_and_goals}
% description of the problem / idea

% goals of the thesis:
    % Main: identify genes that are relevant for research in lung cancer
    % to reach this I will:
        % 1. create a graph database with gene and protein information → to take advantage of the relationships between genes and proteins
        % 2. use a graph algorithm → to identify relevant genes for distinguishing between healthy and lung cancer
        % 3. compare identified genes with other papers → to validate the results

% TODO zum Schluss wenn ich weiß was genau wo drin ist
\subsection{Structure}  \label{subsec:structure}
% TODO K - Chapter or Section?
The thesis starts with a closer look at the biological and computational background of the project (Section~\ref{sec:background}),
which includes a comprehensive overview of lung cancer, gene expression analysis, and graph databases in cancer research.

Next, we describe the experimental setup and methodology employed to collect and analyze data (Section~\ref{sec:experimental-setup}),
including the creation of a graph database and the application of a graph algorithm.

The result of our investigation is presented in the fourth chapter~\ref{sec:results}.

We then delve into the implications and significance of these findings,
providing a comprehensive analysis of their relevance to lung cancer research (Section~\ref{sec:discussion}).

Finally, we conclude by summarizing the key takeaways from our study and provide an outlook
on future research directions (Section~\ref{sec:conclusion}).
% TODO passt die Struktur noch?


\subsection{Limitations} \label{subsec:limitations}
% TODO - K fehlt noch was?
% TODO - Move to Discussion?

% our Choice
Firstly, the scope of our study is focused on lung cancer, which may not be representative of other types of cancers or diseases.

Secondly, we only considered protein-coding genes in our analysis.
This decision was made due to our experimental setup.

% limitied due to dataset
We relied on gene expression data from the GTEx and Cell Model Passport datasets.
While these resources are widely used and well-established, they do have limitations in terms of their scope and coverage.
The GTEx dataset consists of only adult postmortem donor samples, with 2/3 of them between 50 and 69 years old and 2/3 being male.
The Cell Model Passport dataset contains data from donors who are predominantly male (60\%)
and have an age range that is biased towards older adults (50-69 years old).
Especially the focus on the older age range may affect the generalizability of our findings to other populations.\\


x
{\color{lightgray}
In this chapter we talked about the motivation for the work, the goals of the thesis,
the structure of the thesis and the limitations of the work.
}

\section{Background} \label{sec:background}
In this chapter we will take a closer look at the biological and computational background of our project.
Also, we will discuss related work in the field cancer research with graph databases.

\subsection{Biological background} \label{subsec:biological_background}
% TODO Quelle for TPM and Gene expression
% TODO ENS ID Erklären

% Cancer overview
\textbf{Cancer} is a group of diseases characterized by uncontrolled cell growth.
According to research, widespread metastases are the primary cause of death from cancer\cite{who_cancer_fact_sheet}.
There are over 100 different types of cancer\cite{nci_cancer_types}.
The five most common forms of cancer worldwide in 2022 were lung cancer ($>2.4$ million),
breast cancer ($>2.2$ million), colorectal cancer ($>1.9$ million), prostate cancer ($>1.4$ million) and
stomach cancer ($>0.9$ million).
In 2022, cancer was one of the leading causes of death globally,
with approximately 10 million fatalities\cite{ferlay2024global}.

It can be caused by a combination of genetic and environmental factors,
such as diet, radiation, age, exposure to certain chemicals or viruses\cite{nci_cancer_risk}.
Early detection and the right treatment can significantly improve the chances of curing many types of cancer.
\\

% lung cancer
\textbf{Lung cancer} is a type of cancer that affects the lung organ.
There are two main subtypes: non-small cell carcinoma (NSCLC) and small cell carcinoma (SCLC)\cite{nci_lung_cancer_types}.
The primary risk factor for developing lung cancer is smoking,
but passive smoking and environmental pollution also significantly increase the risk\cite{nci_lung_cancer_types}.

Unfortunately, symptoms of lung cancer often resemble those of common colds or other minor illnesses,
such as coughing and fatigue.
This makes it difficult to diagnose until the disease has progressed to an advanced stage\cite{who_lung_cancer}.
\\

% genetical changes in cancer
\textbf{Gene expression} is a crucial factor in the investigation of cancer.
It describes the process by which genes are read and utilized within a cell to synthesize proteins
that regulate cellular growth and other essential processes.
% TODO J - Better formulation - sehr schwammig
Alterations in gene expression can contribute to the uncontrolled multiplication and abnormal behavior of cancer cells,
which ultimately leads to the development and progression of cancer.
\\

\textbf{Transcripts per Million (TPM)} is a measure of gene expression in a cell.
% It indicates the number of RNA copies of a particular gene per million total transcribed RNA molecules present in a sample.
A higher TPM value signifies that the corresponding gene is actively expressed and, consequently, produces more protein.
Through this method, it is possible to determine precisely which genes are activated within a cell and
their involvement in cancer development.

%TODO Multi Omics Data?!

\subsection{Computational background} \label{subsec:computational_background}
% Graph databases
\textbf{Graph databases} are an effective way to model and analyze graph-like data,
which is in contrast to traditional relational databases well-suited for applications involving relationships between entities.
For instance, social networks and biological systems can be effectively represented using these graph structures~\cite{graph_db_survey}.
In particular, graph databases have been widely used in biological data analysis, such as modeling Protein-Protein-Interaction (PPI) networks
that illustrate the interaction between proteins to control cellular processes.
In this thesis, we use a graph database to put our collected cancer data into a meaningful context.

Just like a graph, a graph database consists of two basic components: nodes and edges between nodes.
These components form the foundation for storing and analyzing large amounts of data.

Nodes are the units that store the basic information of a certain type of entity~\cite{graph_db_survey}.
In our case, genes and proteins are represented as two different node types in the database.
Each gene is, therefore, a node with properties, such as an $ID$, a $name$ or a $TPM$ value for its activity.

Relationships between two nodes are represented by an edge, which also can be of different types.
In our graph database, for example, there are $interactions$ as edges between proteins and $connections$ as edges between a gene and a protein.
As in common graph theory, edges can represent structures like one-to-many or many-to-many relationships.
Edges can also have weights, directions, or properties like $name$ or $type$~\cite{graph_db_power_limitations}.
We do not use these additional metadata in our setup, but they could be used to model more complex relationships.

We will implement our graph database using Neo4J, since it is a widely used graph database management system.
\\

\textbf{Graph database algorithms} enable the efficient analysis and extraction of meaningful insights from large-scale network data.
They can be broadly categorized into three primary groups, which provide different perspectives on graph structure.

Traversal and Pathfinding Algorithms enable the identification of shortest or
most optimal paths between nodes in the graph, thereby facilitating the exploration of network topology.
Examples include Depth-First-Search, Breadth-First-Search, Shortest Path, and Max-Flow-Min-Cut~\cite{neo4j_graph_algorithms}.

Centrality Algorithms are instrumental in understanding which nodes within a graph hold significant importance.
By evaluating centrality measures such as degree centrality, closeness centrality, betweenness centrality,
and PageRank, valuable insights into the underlying network structure is gained~\cite{neo4j_graph_algorithms}.

Community Detection Algorithms, also known as clustering or partitioning algorithms,
are essential for identifying groups of nodes within a graph that share similar characteristics
or exhibit extensive connectivity.
Examples include Label Propagation, Louvain Modularity, and Strongly Connected Components~\cite{neo4j_graph_algorithms}.
\\


% PageRank algorithm
In our analysis, we utilize the \textbf{PageRank algorithm} to identify influential nodes within the network
based on their connectivity and centrality.
The PageRank algorithm was originally a method for evaluating and prioritizing websites on the internet.
It was developed by Larry Page and Sergey Brin in 1988 at Stanford University~\cite{page1999pagerank},
the founders of Google.
The general idea behind the algorithm was that the more links pointing to a website, the more important it is.
This concept is extended in the PageRank algorithm, which assigns a node's score by iteratively considering not only its direct connections
but also the indirect relationships through its linked neighbors.

The algorithm calculates a PageRank score per node using the following formula:

\begin{align*}
PR(A)=(1-d)+d(\frac{PR(T_1)}{C(T_1)}+\dots+\frac{PR(T_n)}{C(T_n)})
\end{align*}

where $PR(A)$ is the PageRank score of node $A$,
$T_1$ to $T_n$ are nodes with edges to $A$,
$d$ is the damping factor,
and $C(T_1)$ represents the number of edges to node $T_1$~\cite{neo4j_graph_algorithms}.


\subsection{Related work} \label{subsec:related_work}
The investigation of tumors and the identification of biomarkers for diagnosis and treatment are crucial tasks in oncology. %Quelle4
Over the past few years, various approaches have been explored to address these challenges.
Notably, graph-based methods have gained popularity for analyzing genetic data.
This section presents a brief overview of studies in this field from the last years,
focusing on the use of graph databases and their algorithms for biological applications and cancer research.
By reviewing existing literature, we aim to provide a comprehensive understanding of the current state of the art in this area.

% TODO Choose different Cite styles

% A systematic review of graph-based explorations of PPI networks: methods, resources, and best practices
The recent systematic review by Rout et al. (2024) \cite{rout2024systematic} provides an extensive overview of
various graph-based methodologies for analyzing Protein-Protein Interaction networks,
emphasizing best practices and common challenges in integrating multi-omics data. %TODO eklären?
The authors highlight the importance of graph databases in storing and querying large-scale biological networks
and essential graph algorithms such as centrality measures and community detection that are relevant to my work using PageRank.
% gute Basis für mein Projekt, da sie die wichtigsten Graph-Algorithmen und Best Practices für die Analyse von PPI-Netzwerken hervorheben


% Applying Graph Database Technology for Analyzing Perturbed Co-Expression Networks in Cancer
% alles sehr ähnlich aber Aufbau vom Netzwerk sehr unterschiedlich
A study by Simpson et al. (2020)~\cite{simpson2020applying} investigated the molecular mechanisms underlying various cancer types
through the analysis of gene expression data sourced from the TCGA database.
The authors employed co-expression networks, using Pearson correlation to establish relationships between genes based on their expression patterns.
However, our thesis will take a distinct approach by concentrating specifically on lung cancer
and utilizing PPI networks instead of co-expression networks.
% Additionally, we have opted for TPM (Transcripts Per Million) as the gene expression metric, differing from FPKM used in their study.
In contrast to Simpson et al.'s comprehensive application of multiple graph algorithms, including PageRank,
Louvain community detection and Dijkstra's algorithm in each cancer type, we will focus on a more targeted approach.
Specifically, we will apply the Pagerank algorithm to identify key genes within our PPI networks.


% Network-based prioritization of cancer genes by integrative ranks from multi-omics data
% Netzwerk aufbau ist ähnlich, aber Datenquelle und Zielsetzung sind unterschiedlich
Shang and Liu (2020) \cite{shan2020network} proposed a method for prioritizing cancer genes called iRank,
which integrates various biological levels, including gene and protein expression, as well as Protein-Protein Interactions,
to identify hepatocellular carcinoma.
In contrast, we will focus on building PPI networks to analyze the interactions between proteins and their corresponding genes.
Unlike Shang and Liu's approach, which concentrates on the TCGA dataset, our work will utilize the Cell Model Passport dataset.
Similar to their approach, we will employ the PageRank algorithm to identify important genes in our analysis.

% TODO Abschluss
In summary, the studies mentioned above provide valuable insights into the application of graph-based methods for analyzing biological data.
Each with a slightly different focus, they demonstrate the versatility of graph databases and algorithms in cancer research,
so as we will do in our work.\\


Overall, in the course of this background section,
we have gained a comprehensive overview of the biological background of lung cancer and
the importance of gene expression analysis.
In particular, we have looked at TPM values, which are an important measure of gene activity.

We have established that the use of graph databases is an effective way to analyze complex networks,
and we have explained corresponding algorithms, especially the PageRank algorithm.

Finally, we have reviewed related work in the field of graph databases in cancer research.


\section{Experimental Setup} \label{sec:experimental-setup}
% → rename Methodology or Implementation
\subsection{Data} \label{subsec:data}

\subsection{Methodology} \label{subsec:methodology}




\section{Results} \label{sec:results}
% focus on the results, not on the analysis


\section{Discussion} \label{sec:discussion}

% analysius of the results eg Top 10 Pagerank genes
\subsection{Analysis} \label{subsec:analysis}


% what went well / what went wrong
\subsection{What went well} \label{subsec:what-went-well}

% TODO discussion oder conclusion?
% what can be improved / future work
\subsection{Future Work} \label{subsec:future_work}
{\color{lightgray}
* Pagerank is a good way to find important genes - well established / known / typical research \\
* Clear finding of new genes that may be biomarkers\\
* Analyse the top 10 genes seems to find good results\\

% Future Work
* These should be further investigated in future studies\\
*  collective behavior of multiple genes rather than focusing solely on individual genes
}


\section{Conclusion} \label{sec:conclusion}

% Breif summary of the findings



Content:

preparation of the Datasets for the graph db
* for every dataset we need the format genes as rows and at least the ENS ID and a mean tpm for every gene row
    * GTEX (Genotype-Tissue Expression) - healthy tissues
        * downloaded bulk-tissue-expression file (... more about the data)
        * format was: Gene / ENS ID as rows and tissues as columns - tpm value for every gene in every tissue
        * melted down to genes as rows and tissue and tpm value as columns
        * grouped by tissues to get a single tpm value for every gene (eigentlich sum and count to get mean)
    → 56.156 human genes with mean values for healthy tpm

    * CMP (Cell Model Passport) - cancerous tissues
        * downloaded all RNA-Seq processed data - which contain data from Sanger Institute and from Broad Institute
        * format was: genes with CMP IDs as rows and tpms, tissues (called model), ... per column
        * separate file loaded with model annotation to get a list of all models (tissues) that have lung cancer as cancer type
        * filtered the data to get only the lung cancer models
        * grouped all models (tissues) to get a mean tpm value for every gene
        * needed to add ENS ID to the genes for further processing
        * separate ensemble file downloaded from biomart (https://www.ensembl.org/biomart/martview/)
        * contains Gene stable ID and gene symbol (name)
        * idea is to add the Gene stable ID by merging both tables by the gene name
        * daher gehen wir davon aus, dass jeder gen name eindeutig ist
        * war es nicht - 10.605/48.311 rows in the ensemble file had a not unique gene name
        * rows with names that are not unique were removed
        * merging the ensemble file with the CMP data to get the ENS ID for every gene
        * 3.760 / 37.262 genes had no ENS ID → missing IDs could be added by synonymes maybe because of duplicate removement
        * these genes are removed
    → 33.502 human genes with mean values tpm for lung cancer

    * HPA (Human Protein Atlas) - cancerous tissues
        * downloaded RNA HPA cell line cancer gene data from the HPA website
        * format was: genes with ENS ID as rows and Gene Names, Cancertype and TPM value as columns
        * data filtered for lung cancer
    → 20.162 human genes with tpm values for lung cancer
    → DELETED!!!

Datasets to nodes and edges
    * genes as nodes
        * eigenschaft: name, gene stable id, norm healyth tpm, norm cancerous tpm, delta tpm, z score, delta tpm relevant
        * merging the GTEx (healthy) and CMP (cancerous) data to get a list of all genes that are in both datasets
        * normalizing the TPM Values with log scaling
        * calculate a delta TPM value for every gene as the difference between the mean tpm value in the healthy and the cancerous dataset
        * NOT THE ABSOLUTE
        * to find out which genes have a relevant change in the healthy and cancerous tpm value we calculate the z score for the delta tpm values for every gene
        * z score is calculated as the difference between the delta tpm value and the mean delta tpm value divided by the standard deviation of the delta tpm values
        * z score is a measure of how many standard deviations an element is from the mean
        * 1.643(?!) genes with a z score of 1,96 or higher are considered as delta tpm relevant (p Wert damit 0,05 and Konfidenzniveau 95%)
    → 32.865 gene Nodes

    * protein gene connection as edges
        * eigenschaft: gene stable id, protein stable id
        * proteins are needed as a second node type to get a connection between the genes.
        * downloaded protein file from biomart with gene stable ID and protein stable ID
        * filtered for those gene stable IDs that are in the gene node list.
        * we only have those proteins that have a connection (was sagt die tabelle genau aus?)
    → 101.731 protein gene edges (means: every protein is connected to exactly one gene, but every gene can have multiple proteins)

    * proteins as nodes
        * eigenschaft: protein stable id
        * use the protein gene table as base
        * only use the column with the protein stable ID
    → 101.731 protein Nodes

    * protein protein connection as edges
        * eigenschaft: protein stable id 1, protein stable id 2
        * downloaded protein protein interaction file from STRING database
        * filtered for those protein stable IDs that are in the protein node list
    → 11.247.242 protein protein edges
    (not every protein is connected to another one → but since every gene could have multiple proteins, the connected gene may have a connection to another protein)
    TODO: are those genes that are not connected to another gene relevant for the analysis?

Graph database
    * Neo4J
    * gene nodes query:
    * protein nodes query:
    * gene protein edges query:
    * protein protein edges query:




ANALYSIS
IN ONENOTE


\newpage
\bibliographystyle{plain} % We choose the "plain" reference style
\bibliography{refs} % Entries are in the refs.bib file

\end{document}
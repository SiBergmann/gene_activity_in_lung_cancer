\documentclass[11pt,a4paper]{article}

\usepackage[margin=2cm]{geometry}
\usepackage[utf8]{inputenc}
\usepackage[english]{babel}
\usepackage{amsmath}
\usepackage{amsfonts}
\usepackage{amssymb}
\usepackage{graphicx}
\usepackage{hyperref}

%added userpackages
\usepackage{microtype}
\usepackage{natbib}
\usepackage{lmodern}
\usepackage{bmpsize-base}
\usepackage{xcolor}
\usepackage{hhline}
\setlength{\parindent}{0em}

%%%%%%%%%%%%%%%%%%%%%%%%%

\title{Graph-Based Analysis of Gene Activity in Lung Cancer based on Protein-Interaction Networks}
\author{
Simone Bergmann, 3957624 \\
\texttt{simone.bergmann@uni-bielefeld.de}%
}
%%%%%%%%%%%%%%%%%%%%%%%%%
% \twocolumn
\begin{document}
\maketitle

% FACTS:
% 15 bis 30 Seiten (exklusive Titel, Abstract, Inhaltsverzeichnis, Literaturverzeichnis, ...)
% Deckblatt
% Inhaltsverzeichnis
% Nicht zu viele Unterkapitel / zu sehr verschachtelt
% ca 30 Quellen
% Jedes Kapitel am Anfang und am Ende zusammenfassen aber Fokus auf Ende


% Task:
% Creation of a data model and integration of genes and proteins with interactions in an appropriate form within a graph database,
% which is to be used for the identification of relevant genes for distinguishing between healthy tissue and lung cancer.


% TODO
% Abstact soll kursiv sein
% check tenses in the text - present tense or past tense

% Clean up Code:
    % Write Intro for every jupyter file
    % Check for Comments
    % clear Read Me file
    % Check for unused code
    % Eine py zum ausführen von allem Code


\begin{abstract} \label{abstract}

    Lung cancer remains one of the leading causes of mortality worldwide,
    with significant implications for public health and individual well-being.
    To address this challenge,
    we aim to contribute to the ongoing efforts to combat lung cancer by identifying potential biomarkers using graph databases and algorithms.
    Traditional methods often struggle to capture the complex relationships between genes and proteins.
    We try to overcome these challenges with a graph based approach.
    By analyzing gene expression data and applying a network-based approach,
    we have successfully identified 10 genes that may serve as biomarkers for early detection or personalized treatment.
    Our method integrates over 17,000 genes, 100,000 proteins, and 13,000,000 interactions,
    allowing us to pinpoint relevant genes with high connectivity in the network.
    This approach has revealed promising new targets for further research and
    underscore the importance of considering gene interactions in cancer development.
    While our study provides valuable insights into the genetic landscape of lung cancer,
    it is essential to note that further validation and refinement of our methodology are necessary to ensure the accuracy and reliability of our findings.

\end{abstract}

\section{Introduction} \label{sec:intro}
% TODO Schicker formulieren weil erster Satz der Arbeti oder weglassen
% In the Introduction, the motivation for the work is presented, followed by the goals of the thesis.
%The structure of the thesis is outlined and the limitations of the work are discussed.

% TODO könnte noch etwas länger sein
\subsection{Motivation} \label{subsec:motivation}
% 300 Words

% Problem
Lung cancer is a major public health problem worldwide, caused by various factors,
including smoking, air pollution, and also genetic factors.
% TODO J - Quelle
It is the leading cause of cancer-related deaths worldwide\cite{ferlay2024global}.
Despite advances in medical care, the survival rate for patients remains relatively low,
with only $30.2\%$ of women and $22.1\%$ of men (2016)
surviving beyond five years after diagnosis~\cite{seer2024explorer}, often due to late-stage detection.
Therfore, early detection is crucial to initiate timely treatment and ultimately reduce the mortality rate.
\\
% TODO J - Übergang

% Solution
We focus on analyzing gene expression data to identify potential biomarkers for lung cancer.
We use graph databases for this purpose,
as they offer ways to represent complex relationships between large sets of genes and proteins
and analyze them efficiently using graph based algorithms.
These data provide valuable insights into the underlying molecular mechanisms of lung cancer
and help to improve the early diagnosis and treatment of lung cancer.

\subsection{Problem Statement and Goals} \label{subsec:problem_statement_and_goals}
% description of the problem / idea

% goals of the thesis:
    % Main: identify genes that are relevant for research in lung cancer
    % to reach this I will:
        % 1. create a graph database with gene and protein information → to take advantage of the relationships between genes and proteins
        % 2. use a graph algorithm → to identify relevant genes for distinguishing between healthy and lung cancer
        % 3. compare identified genes with other papers → to validate the results

% TODO zum Schluss wenn ich weiß was genau wo drin ist
\subsection{Structure}  \label{subsec:structure}
% TODO K - Chapter or Section?
The thesis starts with a closer look at the biological and computational background of the project (Section~\ref{sec:background}),
which includes a comprehensive overview of lung cancer, gene expression analysis, and graph databases in cancer research.

Next, we describe the experimental setup and methodology employed to collect and analyze data (Section~\ref{sec:experimental-setup}),
including the creation of a graph database and the application of a graph algorithm.

The result of our investigation is presented in the fourth chapter~\ref{sec:results}.

We then delve into the implications and significance of these findings,
providing a comprehensive analysis of their relevance to lung cancer research (Section~\ref{sec:discussion}).

Finally, we conclude by summarizing the key takeaways from our study and provide an outlook
on future research directions (Section~\ref{sec:conclusion}).
% TODO passt die Struktur noch?


\subsection{Limitations} \label{subsec:limitations}
% TODO - K fehlt noch was?
% TODO - Move to Discussion?

% our Choice
Firstly, the scope of our study is focused on lung cancer, which may not be representative of other types of cancers or diseases.

Secondly, we only considered protein-coding genes in our analysis.
This decision was made due to our experimental setup.

% limitied due to dataset
We relied on gene expression data from the GTEx and Cell Model Passport datasets.
While these resources are widely used and well-established, they do have limitations in terms of their scope and coverage.
The GTEx dataset consists of only adult postmortem donor samples, with 2/3 of them between 50 and 69 years old and 2/3 being male.
The Cell Model Passport dataset contains data from donors who are predominantly male (60\%)
and have an age range that is biased towards older adults (50-69 years old).
Especially the focus on the older age range may affect the generalizability of our findings to other populations.\\


x
{\color{lightgray}
In this chapter we talked about the motivation for the work, the goals of the thesis,
the structure of the thesis and the limitations of the work.
}

\section{Background} \label{sec:background}
In this chapter we will take a closer look at the biological and computational background of our project.
Also, we will discuss related work in the field cancer research with graph databases.

\subsection{Biological background} \label{subsec:biological_background}
% TODO Quelle for TPM and Gene expression
% TODO ENS ID Erklären

% Cancer overview
\textbf{Cancer} is a group of diseases characterized by uncontrolled cell growth.
According to research, widespread metastases are the primary cause of death from cancer\cite{who_cancer_fact_sheet}.
There are over 100 different types of cancer\cite{nci_cancer_types}.
The five most common forms of cancer worldwide in 2022 were lung cancer ($>2.4$ million),
breast cancer ($>2.2$ million), colorectal cancer ($>1.9$ million), prostate cancer ($>1.4$ million) and
stomach cancer ($>0.9$ million).
In 2022, cancer was one of the leading causes of death globally,
with approximately 10 million fatalities\cite{ferlay2024global}.

It can be caused by a combination of genetic and environmental factors,
such as diet, radiation, age, exposure to certain chemicals or viruses\cite{nci_cancer_risk}.
Early detection and the right treatment can significantly improve the chances of curing many types of cancer.
\\

% lung cancer
\textbf{Lung cancer} is a type of cancer that affects the lung organ.
There are two main subtypes: non-small cell carcinoma (NSCLC) and small cell carcinoma (SCLC)\cite{nci_lung_cancer_types}.
The primary risk factor for developing lung cancer is smoking,
but passive smoking and environmental pollution also significantly increase the risk\cite{nci_lung_cancer_types}.

Unfortunately, symptoms of lung cancer often resemble those of common colds or other minor illnesses,
such as coughing and fatigue.
This makes it difficult to diagnose until the disease has progressed to an advanced stage\cite{who_lung_cancer}.
\\

% genetical changes in cancer
\textbf{Gene expression} is a crucial factor in the investigation of cancer.
It describes the process by which genes are read and utilized within a cell to synthesize proteins
that regulate cellular growth and other essential processes.
% TODO J - Better formulation - sehr schwammig
Alterations in gene expression can contribute to the uncontrolled multiplication and abnormal behavior of cancer cells,
which ultimately leads to the development and progression of cancer.
\\

\textbf{Transcripts per Million (TPM)} is a measure of gene expression in a cell.
% It indicates the number of RNA copies of a particular gene per million total transcribed RNA molecules present in a sample.
A higher TPM value signifies that the corresponding gene is actively expressed and, consequently, produces more protein.
Through this method, it is possible to determine precisely which genes are activated within a cell and
their involvement in cancer development.

%TODO Multi Omics Data?!

\subsection{Computational background} \label{subsec:computational_background}
% Graph databases
\textbf{Graph databases} are an effective way to model and analyze graph-like data,
which is in contrast to traditional relational databases well-suited for applications involving relationships between entities.
For instance, social networks and biological systems can be effectively represented using these graph structures~\cite{graph_db_survey}.
In particular, graph databases have been widely used in biological data analysis, such as modeling Protein-Protein-Interaction (PPI) networks
that illustrate the interaction between proteins to control cellular processes.
In this thesis, we use a graph database to put our collected cancer data into a meaningful context.

Just like a graph, a graph database consists of two basic components: nodes and edges between nodes.
These components form the foundation for storing and analyzing large amounts of data.

Nodes are the units that store the basic information of a certain type of entity~\cite{graph_db_survey}.
In our case, genes and proteins are represented as two different node types in the database.
Each gene is, therefore, a node with properties, such as an $ID$, a $name$ or a $TPM$ value for its activity.

Relationships between two nodes are represented by an edge, which also can be of different types.
In our graph database, for example, there are $interactions$ as edges between proteins and $connections$ as edges between a gene and a protein.
As in common graph theory, edges can represent structures like one-to-many or many-to-many relationships.
Edges can also have weights, directions, or properties like $name$ or $type$~\cite{graph_db_power_limitations}.
We do not use these additional metadata in our setup, but they could be used to model more complex relationships.

We will implement our graph database using Neo4J, since it is a widely used graph database management system.
\\

\textbf{Graph database algorithms} enable the efficient analysis and extraction of meaningful insights from large-scale network data.
They can be broadly categorized into three primary groups, which provide different perspectives on graph structure.

Traversal and Pathfinding Algorithms enable the identification of shortest or
most optimal paths between nodes in the graph, thereby facilitating the exploration of network topology.
Examples include Depth-First-Search, Breadth-First-Search, Shortest Path, and Max-Flow-Min-Cut~\cite{neo4j_graph_algorithms}.

Centrality Algorithms are instrumental in understanding which nodes within a graph hold significant importance.
By evaluating centrality measures such as degree centrality, closeness centrality, betweenness centrality,
and PageRank, valuable insights into the underlying network structure is gained~\cite{neo4j_graph_algorithms}.

Community Detection Algorithms, also known as clustering or partitioning algorithms,
are essential for identifying groups of nodes within a graph that share similar characteristics
or exhibit extensive connectivity.
Examples include Label Propagation, Louvain Modularity, and Strongly Connected Components~\cite{neo4j_graph_algorithms}.
\\


% PageRank algorithm
In our analysis, we utilize the \textbf{PageRank algorithm} to identify influential nodes within the network
based on their connectivity and centrality.
The PageRank algorithm was originally a method for evaluating and prioritizing websites on the internet.
It was developed by Larry Page and Sergey Brin in 1988 at Stanford University~\cite{page1999pagerank},
the founders of Google.
The general idea behind the algorithm was that the more links pointing to a website, the more important it is.
This concept is extended in the PageRank algorithm, which assigns a node's score by iteratively considering not only its direct connections
but also the indirect relationships through its linked neighbors.

The algorithm calculates a PageRank score per node using the following formula:

\begin{align*}
PR(A)=(1-d)+d(\frac{PR(T_1)}{C(T_1)}+\dots+\frac{PR(T_n)}{C(T_n)})
\end{align*}

where $PR(A)$ is the PageRank score of node $A$,
$T_1$ to $T_n$ are nodes with edges to $A$,
$d$ is the damping factor,
and $C(T_1)$ represents the number of edges to node $T_1$~\cite{neo4j_graph_algorithms}.


\subsection{Related work} \label{subsec:related_work}
The investigation of tumors and the identification of biomarkers for diagnosis and treatment are crucial tasks in oncology. %Quelle4
Over the past few years, various approaches have been explored to address these challenges.
Notably, graph-based methods have gained popularity for analyzing genetic data.
This section presents a brief overview of studies in this field from the last years,
focusing on the use of graph databases and their algorithms for biological applications and cancer research.
By reviewing existing literature, we aim to provide a comprehensive understanding of the current state of the art in this area.

% TODO Choose different Cite styles

% A systematic review of graph-based explorations of PPI networks: methods, resources, and best practices
The recent systematic review by Rout et al. (2024) \cite{rout2024systematic} provides an extensive overview of
various graph-based methodologies for analyzing Protein-Protein Interaction networks,
emphasizing best practices and common challenges in integrating multi-omics data. %TODO eklären?
The authors highlight the importance of graph databases in storing and querying large-scale biological networks
and essential graph algorithms such as centrality measures and community detection that are relevant to my work using PageRank.
% gute Basis für mein Projekt, da sie die wichtigsten Graph-Algorithmen und Best Practices für die Analyse von PPI-Netzwerken hervorheben


% Applying Graph Database Technology for Analyzing Perturbed Co-Expression Networks in Cancer
% alles sehr ähnlich aber Aufbau vom Netzwerk sehr unterschiedlich
A study by Simpson et al. (2020)~\cite{simpson2020applying} investigated the molecular mechanisms underlying various cancer types
through the analysis of gene expression data sourced from the TCGA database.
The authors employed co-expression networks, using Pearson correlation to establish relationships between genes based on their expression patterns.
However, our thesis will take a distinct approach by concentrating specifically on lung cancer
and utilizing PPI networks instead of co-expression networks.
% Additionally, we have opted for TPM (Transcripts Per Million) as the gene expression metric, differing from FPKM used in their study.
In contrast to Simpson et al.'s comprehensive application of multiple graph algorithms, including PageRank,
Louvain community detection and Dijkstra's algorithm in each cancer type, we will focus on a more targeted approach.
Specifically, we will apply the Pagerank algorithm to identify key genes within our PPI networks.


% Network-based prioritization of cancer genes by integrative ranks from multi-omics data
% Netzwerk aufbau ist ähnlich, aber Datenquelle und Zielsetzung sind unterschiedlich
Shang and Liu (2020) \cite{shan2020network} proposed a method for prioritizing cancer genes called iRank,
which integrates various biological levels, including gene and protein expression, as well as Protein-Protein Interactions,
to identify hepatocellular carcinoma.
In contrast, we will focus on building PPI networks to analyze the interactions between proteins and their corresponding genes.
Unlike Shang and Liu's approach, which concentrates on the TCGA dataset, our work will utilize the Cell Model Passport dataset.
Similar to their approach, we will employ the PageRank algorithm to identify important genes in our analysis.

% TODO Abschluss
In summary, the studies mentioned above provide valuable insights into the application of graph-based methods for analyzing biological data.
Each with a slightly different focus, they demonstrate the versatility of graph databases and algorithms in cancer research,
so as we will do in our work.\\


Overall, in the course of this background section,
we have gained a comprehensive overview of the biological background of lung cancer and
the importance of gene expression analysis.
In particular, we have looked at TPM values, which are an important measure of gene activity.

We have established that the use of graph databases is an effective way to analyze complex networks,
and we have explained corresponding algorithms, especially the PageRank algorithm.

Finally, we have reviewed related work in the field of graph databases in cancer research.


\section{Experimental Setup} \label{sec:experimental-setup}
% → rename Methodology or Implementation

\subsection{Data Retrieval and Preprocessing} \label{subsec:data}

For our study we choose two main datasets, each containing gene expression data for healthy and cancerous tissues.
The first one includes healthy tissue data from the Genotype-Tissue Expression (GTEx) project,
and the second one includes cancerous tissue data from the Cell Model Passport (CMP) project.
While these resources are widely used and well-established in research, they do have limitations in terms of their scope and coverage.
The GTEx dataset consists of only adult postmortem donor samples, with 2/3 of them between 50 and 69 years old and 2/3 being male~\cite{GTEX_modelannotation}.
The Cell Model Passport dataset contains data from donors who are predominantly male (60\%)
and have an age range that is biased towards older adults (50-69 years old)~\cite{CMP_modelannotation}.

The two datasets provide the foundation for our graph database, specifically for the gene nodes.
We aim to create a table with genes as rows,
where each row contains a unique Ensemble ID along with one value representing healthy TPM and another value representing cancerous TPM.
\\


\subsubsection*{GTEx} \label{subsubsec:GTEx}
For the healthy tissue samples used in this study,
we utilized the \texttt{GTEx\_Analysis\_2017-06-05\_v8\_\newline
RNASeQCv1.1.9\_gene\_tpm} dataset from the GTEx portal~\cite{gtex_download}.
The \textbf{GTEx portal} is a large-scale, publicly available resource for studying gene activity.
The Adult GTEx project aims to characterize the gene expression patterns in healthy tissues across different individuals,
providing valuable insights into the underlying biology of human development and disease.

The used dataset $A_{orig}$ is in a .gct file format that contains TPM values for 56,156 genes $i$ identified by
Ensemble ID as rows in 17,382 different tissues $j$ as columns (see~\cref{tab:gtex_table}.0).
The data was initially stored in a file in wide format where each tissue $j$ had its own column.
The TPM values for these tissues range between 0 and 747,400.
Since there are no missing values in the dataset, we did not need to handle any missing data.
To process this data into a suitable format, we employed the following steps:

\begin{enumerate}
    \item \textbf{Reshaping to long format:}
    We read the data from the original format in chunks of 3,000 rows at a time due to RAM capacity constraints.
    For each chunk of data, we transformed the columns for each tissue $j$ into individual rows,
    resulting in a dataset with three columns and 976.103.592 ($i * j$) rows (see~\cref{tab:gtex_table}.1).

    \item \textbf{Grouping by genes:}
    Once all chunks had been processed, we separated the combined dataset again for RAM reasons
    in new chunks of approximately 200 million rows.
    These chunks have been grouped by gene using an aggregate function that calculated
    both the sum $S_i$ and count $C_i$ of TPM values for each gene $i$ (see~\cref{tab:gtex_table}.2).

    \item \textbf{Calculating mean TPM:}
    To handle genes that had been split across multiple chunks,
    we performed a global aggregation on the genes of the sum of the dataset.
    Then we calculated the mean TPM value $M_i$ for each gene $i$ by dividing the sum $S_i$ by the count $C_i$
    of each observation (see~\cref{tab:gtex_table}.3).
\end{enumerate}

\begin{table}[h]
    \centering
    \resizebox{\textwidth}{!}{
    \begin{tabular}{|c|c|c|c|}
        \hline
        \textbf{0. Original Format} & \textbf{1. Reshaping to long format} & \textbf{2. Grouping by genes} & \textbf{3. Calculating mean TPM} \\
        \hline
        & & & \\[1mm] % adding more space to the row
        $\begin{aligned}
        i & : \text{gene} \\
        j & : \text{tissue} \\
        a_{ij} & : \text{TPM for gene } i \text{ in tissue } j
        \end{aligned}$ &
        $ A_{\text{long}} = (\text{Gen}_i, \text{Tissue}_j, a_{ij}) $ &
        $ A_{\text{agg}} = (\text{Gen}_i, S_i, C_i )$ &
        $ A_{\text{mean}} = (\text{Gen}_i, M_i )$ \\

        & & & \\[1mm] % adding more space to the row

        & & $ S_i = \sum_{j=1}^{j} a_{ij}, \quad C_i = \sum_{j=1}^{j} 1 $ &
        $ M_i = \frac{S_i}{C_i} $ \\

        & & & \\[1mm] % adding more space to the row

        $ A_{\text{orig}} = \begin{bmatrix}
            a_{11} & a_{12} & \dots & a_{1j} \\
            a_{21} & a_{22} & \dots & a_{2j} \\
            \vdots & \vdots & \ddots & \vdots \\
            a_{i1} & a_{i2} & \dots & a_{ij}
        \end{bmatrix} $ &
        $ A_{\text{long}} = \begin{bmatrix}
            \text{Gen}_1 & \text{Tissue}_1 & a_{11} \\
            \text{Gen}_1 & \text{Tissue}_2 & a_{12} \\
            \vdots & \vdots & \vdots \\
            \text{Gen}_i & \text{Tissue}_j & a_{ij}
        \end{bmatrix} $ &
        $ A_{\text{agg}} = \begin{bmatrix}
            \text{Gen}_1 & S_{1} & C_{1} \\
            \text{Gen}_2 & S_{2} & C_{2} \\
            \vdots & \vdots & \vdots \\
            \text{Gen}_i & S_{i} & C_{i} \\
        \end{bmatrix} $ &
        $ A_{\text{mean}} = \begin{bmatrix}
            \text{Gen}_1 & M_{1}\\
            \text{Gen}_2 & M_{2} \\
            \vdots & \vdots\\
            \text{Gen}_i & M_{i}\\
        \end{bmatrix} $ \\

        & & & \\[1mm] % adding more space to the row
        \hline
        & & & \\[1mm] % adding more space to the row

        % Example Matrix Representation
        $ A_{\text{orig}, i} = \begin{bmatrix}
            8.764 & 0.07187 & \dots & 3.215
        \end{bmatrix}$ &

        $ A_{\text{long}, i} = \begin{bmatrix}
            \text{ENSG...938} & \text{GTEX-111...} & 8.764 \\
            \text{ENSG...938} & \text{GTEX-112...} & 0.07187 \\
            \text{ENSG...938} & \text{GTEX-113...} & 3.215
        \end{bmatrix}$ &

        $ A_{\text{agg}, i} = \begin{bmatrix}
            \text{ENSG...938} & 12.051 & 3
        \end{bmatrix}$ &

        $ A_{\text{mean}, i} = \begin{bmatrix}
            \text{ENSG...938} & 4.017
        \end{bmatrix}$ \\

        & & & \\[1mm] % adding more space to the row
        \hline
    \end{tabular}
    }
    \caption{Data transformation pipeline for the GTEx dataset: Formular and example data per gene}\label{tab:gtex_table}
\end{table}

The resulting dataset (see~\cref{fig:03_01_df_GTEX_healthy_mean}) with 56,156 genes $i$
and a mean TPM value $M_i$ was saved as a CSV file for further processing.\\

\begin{figure}[h]
    \centering
    \includegraphics[height=\dfheight]{figures/03_01_GTEX_healthy_mean}
    \caption{Example data of processed Genotype-Tissue Expression dataset}
    \label{fig:03_01_df_GTEX_healthy_mean}
\end{figure}



%%%%%%%%%%%%%%%%%%%%%%%%%%%%%%%%%%%%%%%%%%%%%%%%%%%%%%%%%%%%%%%%%%%%%%%%%%%%%%%%%%%%%%%%%%%%
% Cell Model Passport
\subsubsection*{CMP} \label{subsubsec:CMP}
For the purpose of the analysis of gene activity in lung cancer, we utilized data from the \textbf{CMP project},
a comprehensive resource for studying cancer-related gene expression.

We obtained the dataset from the CMP portal~\cite{cmp_download} with the name \texttt{rnaseq\_all\_data\_20220624}.
This dataset contains data from the Sanger Institute and the Broad Institute and
consists of a file containing genes associated with diverse cancer types, including lung cancer.
Initially, the data was stored in long format with columns for gene identifiers, tissues, TPM values,
and additional information.

However, the CMP dataset lacked Ensemble IDs, which were crucial for our analysis.
So we needed to add the Ensemble IDs to the dataset to ensure consistency and accuracy in our analysis.
To focus on lung cancer-specific data, we loaded an additional file containing tissue-specific annotations~\cite{cmp_tissue_models}.
Then we filtered the CMP dataset to include only tissues from the annotation file with lung cancer as the cancer type.
Specifically, we filtered the dataset to include only lung cancer-specific models labeled as
Small Cell Lung Carcinoma, Non-Small Cell Lung Carcinoma or Squamous Cell Lung Carcinoma in the $cancer\_type$ column.

The resulting dataset comprises 7,564,389 rows containing genes $i$ and tissues $j$
with associated TPM values for lung cancer (see~\cref{tab:cmp_table}.0).
Specifically, the dataset includes information on 37,262 unique genes $i$ across 203 distinct tissue types $j$.
The resulting dataset contains no missing values and the TPM values span a range of 0 and 132,676.

To prepare the data for further processing, we performed the following steps:
\begin{enumerate}
    \item \textbf{Grouping by genes:}
    We grouped the dataset by genes to obtain a mean TPM value for every gene.
    This step involved aggregating the data by gene names, resulting in a new dataset
    with a mean TPM value for each gene (see~\cref{tab:cmp_table}.1).

    \item \textbf{Adding Ensembl ID:}
    The original dataset contained own IDs for the genes but lacked the universal Ensembl ID required for matching genes across datasets.
    We overcame this challenge by adding the required Ensembl IDs to our dataset using gene names as a reference point.
    For this purpose we downloaded an Ensemble file from biomart~\cite{bio_marts},
    which contains the Ensembl ID and corresponding name for a gene $x$.

    By analyzing the Ensemble file, we encountered an issue where some gene name were not unique within the file.
    To resolve this problem, we dropped all rows with duplicate gene names.
    We then merged the Ensemble table with our CMP data on the gene name to retrieve the Ensemble ID for each gene (see~\cref{tab:cmp_table}.2).

    \item \textbf{Removing missing Ensembl ID:}
    After merging the data, we found that 3,760 of our 37,262 genes still had no Ensembl ID associated with them.
    Since these genes were likely duplicates or did not exist in the Ensemble file,
    we removed them from our dataset to ensure consistency and accuracy of our analysis (see~\cref{tab:cmp_table}.3).
\end{enumerate}

\begin{table}[!h]
    \centering
    \resizebox{\textwidth}{!}{
    \begin{tabular}{|c|c|c|c|}
        \hline
        \textbf{0. Original Format} & \textbf{1. Grouping by genes} & \textbf{2. Adding Ensembl ID} & \textbf{3. Removing missing Ensembl ID} \\
        \hline
        & & & \\[1mm] % adding more space to the row

        % 1. Additional formulars
        $\begin{aligned}
        i & : \text{gene} \\
        j & : \text{tissue} \\
        b_{ij} & : \text{TPM for gene } i \text{ in tissue } j
        \end{aligned}$ &

        $ M_i = \frac{1}{j}\sum_{j=1}^{j} b_{ij} $ &

        $\text{EID}_i =
        \begin{cases}
            \text{EID}_x & \text{if } \exists x \in E_{\text{ens}}: \text{Name}_x = \text{Name}_i, \\
            \text{NULL} & \text{if } \nexists x \in E_{\text{ens}} : \text{Name}_x = \text{Name}_i.
        \end{cases}$ &

        $ B_{\text{clean}} = B_{\text{ens}} \setminus \{ i \mid EID_i = \text{NULL} \} $
        \\

        & & & \\[1mm] % adding more space to the row

        % 2. Row definitions
        $ B_{\text{orig}} = ( \text{ID}_i, \text{Name}_i, \text{Tissue}_j, b_{ij}, \dots) $ &
        $ B_{\text{agg}} =  ( \text{ID}_i, \text{Name}_i, M_i) $ &
        \makecell{$ E_{\text{ens}} =  ( \text{EID}_x, \text{Name}_x) $ \\
            $ B_{\text{ens}} =  ( \text{ID}_i, \text{Name}_i, \text{EID}_i, M_i) $} &
        $ B_{\text{clean}} =  ( \text{ID}_i, \text{Name}_i, \text{EID}_i, M_i) $
        \\

        & & & \\[1mm] % adding more space to the row


        % 3. Style of the Matrix
        $ B_{\text{orig}} = \begin{bmatrix}
            \text{ID}_1 & \text{Name}_1 & \text{Tissue}_1 & b_{11} \\
            \text{ID}_1 & \text{Name}_1 & \text{Tissue}_2 & b_{12} \\
            \vdots & \vdots & \vdots & \vdots \\
            \text{ID}_i & \text{Name}_i & \text{Tissue}_j & b_{ij}
        \end{bmatrix} $ &
        $ B_{\text{agg}} = \begin{bmatrix}
            \text{ID}_1 & \text{Name}_1 & M_{1} \\
            \text{ID}_2 & \text{Name}_2 & M_{2} \\
            \vdots & \vdots & \vdots \\
            \text{ID}_i & \text{Name}_i & M_{i}
        \end{bmatrix} $ &
        $ B_{\text{ens}} = \begin{bmatrix}
            \text{ID}_1 & \text{Name}_1 & \text{EID}_1 & M_{1} \\
            \text{ID}_2 & \text{Name}_2 & NULL &  M_{2} \\
            \vdots & \vdots & \vdots & \vdots \\
            \text{ID}_i & \text{Name}_i & \text{EID}_i & M_{i}
        \end{bmatrix} $ &
        $ B_{\text{clean}} = \begin{bmatrix}
            \text{ID}_1 & \text{Name}_1 & \text{EID}_1 & M_{1} \\
            \text{ID}_3 & \text{Name}_3 & \text{EID}_3 & M_{3} \\
            \vdots & \vdots & \vdots & \vdots \\
            \text{ID}_i & \text{Name}_i & \text{EID}_i & M_{i}
        \end{bmatrix} $
        \\


        & & & \\[1mm] % adding more space to the row
        \hline
        & & & \\[1mm] % adding more space to the row

        % 4. Example Matrix for a single gene
        $ B_{\text{org}, i} = \begin{bmatrix}
            \text{SIDG...16} & \text{CASP10} & \text{SIDM...13} & 14.41 \\
            \text{SIDG...16} & \text{CASP10} & \text{SIDM...61} & 0.64 \\
            \text{SIDG...16} & \text{CASP10} & \text{SIDM...50} & 3.26
        \end{bmatrix}$ &
        $ B_{\text{agg}, i} = \begin{bmatrix}
            \text{SIDG...16} & \text{CASP10} & 5.017
        \end{bmatrix}$ &
        $ B_{\text{ens}, i} = \begin{bmatrix}
            \text{SIDG...16} & \text{CASP10} & \text{ENSG...400} & 5.017
        \end{bmatrix}$ &
        $ B_{\text{clean}, i} = \begin{bmatrix}
            \text{SIDG...16} & \text{CASP10} & \text{ENSG...400} & 5.017
        \end{bmatrix}$
        \\


        & & & \\[1mm] % adding more space to the row
        \hline
    \end{tabular}
    }
    \caption{Data transformation pipeline for the CMP dataset: Formular and example data per gene}\label{tab:cmp_table}
\end{table}


The resulting dataset (\cref{fig:03_01_df_CMP_cancer_mean}) contains 33,502 genes with mean TPM values for lung cancer and was saved as a CSV file for further processing.

\begin{figure}[h]
    \centering
    \includegraphics[height=\dfheight]{figures/03_01_CMP_cancer_mean}
    \caption{Example data of processed Cell Modell Passport dataset}
    \label{fig:03_01_df_CMP_cancer_mean}
\end{figure}

\subsection{Nodes and Edges} \label{subsec:nodes_and_edges}

As a next step we focus on describing how to create the base information for our advanced PPI Network.
The graph database contains gene and protein node types.
The Edges between the proteins build a classical PPI Network and are called Interactions. % mehr biologische Erklärung?
The second type of Edges, called Connections, are the Link between Proteins and Genes.
As Shown in the Figure \ref{fig:03_02_Network}.
% TODO Wir machen das, weil …

\begin{figure}[h]
\centering
\includegraphics[scale=0.5]{figures/03_02_Network}
\caption{Schema of the Graph Database}
\label{fig:03_02_Network}
\end{figure}

For each of those 4 components we will create a table that will serve as base for creating the graph database.\\


For creating the \textbf{gene nodes} we need to use the preprocessed CMP and GTEx datasets,
which contain mean TPM values for cancerous and healthy genes.
We build the intersection of both datasets on their ENS ID to get a subset that only contains genes with TPM values for both conditions.
% filtering for genes that have a gene-protein connection
To fulfill our first objective \ref{obj:delta_tpm},
we need to calculate a measure that captures significant changes between cancerous and healthy gene activity.
When examining the mean TPM values per dataset, we observe a right-skewed distribution, with most values close to zero
and a long tail extending towards higher values.
The cancerous TPM values vary from 0 to approximately 41.173, while the healthy TPM values range from 0 to around 36.200.



To normalize the TPM values from both datasets and enable better comparability, we perform a common log scaling between 0 and 1 for all TPM values combined.

\begin{equation}
\label{eq:tpm_normalization}
f(x) = \frac{\log(1 + x) - \log(1 + x_{min})}{\log(1 + x_{max}) - \log(1 + x_{min})}
\end{equation}
% TODO Naming log_norm(x) - entsptechend Funktion im Code anpassen

where $x_{max}$ and $x_{min}$ are the maximum and minimum TPM values across both datasets.
After applying the normalization, the distribution of the TPM values is more balanced, as shown in Figure \ref{fig:03_02_normalized_tpm}.

\begin{figure}[h]
\centering
\includegraphics[scale=0.35]{figures/03_02_normalized_tpm}
\caption{Distribution of TPM Values}
\label{fig:03_02_normalized_tpm}
\end{figure}
% TODO Achsenbeschriftung hinzufügen | Title Histogramm


Next, we calculate the difference between the normalized mean healthy and cancerous TPM values per gene
by subtracting the two values and call it $\Delta_{TPM}$.
The distribution of $\Delta_{TPM}$ values is shown in Figure \ref{fig:03_02_delta_tpm}.

\begin{figure}[h]
\centering
\includegraphics[scale=0.5]{figures/03_02_delta_tpm}
\caption{Distribution of $\Delta_{TPM}$ Values}
\label{fig:03_02_delta_tpm}
\end{figure}

We then define a $\Delta_{type}$ as either $increase$ or $decrease$, depending on whether the delta value is positive or negative.

As a final step for our first objective \ref{obj:delta_tpm}, we need to determine if a change in gene activity is significant. % Doppelt
To do this, we use the z score measure, which calculates how many standard deviations a delta TPM value is away
from the mean of all delta TPM values.
The $z score$ is given by:

\begin{subequations}
\begin{equation} \label{eq:z_score}
z score (x) = \frac{x - {\mu}}{\sigma}
\end{equation}
\begin{equation}
\text{where } \mu = \frac{1}{n} \sum_{i=1}^{n} x_i
\end{equation}
\begin{equation}
\text{where } \sigma = \sqrt{\frac{1}{n-1} \sum_{i=1}^{n} (x_i - \mu)^2}
\end{equation}
\end{subequations}
% TODO check formular | mean(x) = and std(x) =

where $x$ is the delta TPM value, $\mu$ is the mean of all delta TPM values, $\sigma$ is the standard deviation of all delta TPM values,

We define a threshold of  $z score = 1.96$ to indicate significant changes in gene activity,
which corresponds to a confidence level of 95\% (p = 0.05).
Genes with delta TPM values exceeding this threshold will be flagged as $true$ in the $\Delta_{tpm} relevant$ column.

The resulting table contains 17.626 Gene Nodes as rows with their associated attributes,
including TPM values and derived metrics such as Delta TPM or Z-Score.
The head of the table is shown in Figure \ref{fig:03_02_gene_nodes}.

\begin{figure}[h]
\centering
\includegraphics[scale=0.5]{figures/03_02_gene_nodes}
\caption{Example data of Gene Nodes Table}
\label{fig:03_02_gene_nodes}
\end{figure}


% Gene Protein Edges
To construct the \textbf{gene-protein edges} we need a table that links the gene to the corresponding protein
which is translated from the transcript of this gene.
For this purpose we downloaded a file from biomart with Gene IDs and their Protein IDs. [LINK]
% The initial dataset comprised of XXX entries.
First we filtered for a subset (Intersection) to only include rows where the Ensembl ID for the gene matched an existing gene node
Since we have some genes without an entry for proteins, we need to drop them ; otherwise, there will not be an edge.

The final gene-protein edge table features 101,731 rows as edges and two columns:
Ensembl ID for the gene and Ensembl ID for the protein translation.
The dataset highlights a key aspect of protein biology - one gene can be translated into multiple proteins.

\begin{figure}[h]
\centering
\includegraphics[scale=0.5]{figures/03_02_gene_protein_edges}
\caption{Example data of Gene Protein Edges Table}
\label{fig:03_02_gene_protein_edges}
\end{figure}


% Proteins Nodes
We can generate the \textbf{protein nodes} from the gene-protein edges
since we only need a PPI for those proteins that are linked to our genes of interest.
To do this, we filter out the Gene colum from the previous file and check if there are any duplicate proteins.
Since there are no duplicate proteins our data indicates that every protein is uniquely translated by a single gene.
We do not need any additional attributes for these protein nodes because are focusing on the edges of this network.

The resulting table is a list of 101.731 unique Protein Ensembl IDs as shown in Figure \ref{fig:03_02_protein_nodes}.
\begin{figure}[h]
\centering
\includegraphics[scale=0.5]{figures/03_02_protein_nodes}
\caption{Example data of Gene Nodes Table}
\label{fig:03_02_protein_nodes}
\end{figure}
\\

% protein protein edges
To create the protein-protein edges we download the String Database [LINK] with the information about the Protein Protein Interaction.
We ensure that there are no duplicate entries.

The resulting file consists of 11.247.242 rows of protein-protein edges with a column for both protein IDs for the edge.
\begin{figure}[h]
\centering
\includegraphics[scale=0.5]{figures/03_02_protein_edges}
\caption{Example data of Gene Nodes Table}
\label{fig:03_02_protein_edges}
\end{figure}




\subsection{Graph Database} \label{subsec:graph_database}
% Checked - all edges are created with 'match' statement

% Intro
As we have created our base data models, our next objective~\ref{obj:graph_algorithm} is to create a graph database
that can be used to perform the PageRank algorithm as a graph algorithm.
With four huge datasets at our disposal, optimizing the generation of the database is crucial.
In this section, we describe the queries, called cypher queries, used to create and populate our graph database,
including information about creating indexes, constraints, relationships between nodes, and loading data into the database.\\


% PPI Network
To construct the \textbf{basic PPI network}, we need to create protein nodes and their interactions as edges.
To optimize query execution, we start with indexing the protein property $ID$.
Next we load the data (\ref{subsubsec:protein_nodes}~-~Protein nodes) as a list and then create the nodes in batches for efficient processing.
The graph database is populated with 104,235 protein nodes.

Since these nodes are used solely for connections between genes, no additional properties are required.
The cypher query for creating protein nodes is:
\begin{lstlisting}[language=Cypher, label={lst:protein_nodes}]
    CREATE (p:protein {id: 'Protein ID'})
\end{lstlisting}
\vspace{\baselineskip}

We then create interaction edges by loading the saved data (\ref{subsubsec:protein_protein_edges}~-~Protein-protein edges)
as a list of protein tuples and searching for both protein nodes by their ID in the graph database.
The edges are created in batches for efficient processing, resulting in 11,247,242 interactions between protein nodes.

The cypher query for creating protein-protein edges is:
\begin{lstlisting}[language=Cypher, label={lst:protein_edges}]
    MATCH (s:protein{id:'left Protein ID'})
    MATCH (s:protein{id:'right Protein ID'})
    CREATE (s)-[:interaction]->(t)
\end{lstlisting}

We now have a complete PPI network in place.\\

% MATCH p=()-[r:interaction]->() RETURN p LIMIT 5
\begin{figure}[h]
    \centering
    \includegraphics[width=0.35\textwidth]{figures/03_03_Basic_Network}
    \caption{Extract from the graph database setup showing example protein nodes and their interactions}
    \label{fig:03_03_Basic_Network}
\end{figure}
\vspace{\baselineskip}


%%%%%%%%%%%%%%%%%%%%%%%%%%%%%%%%%%%%%%%%%%%%%%%%%%%%%%%%%%%%%%%%%%%%%%%%%%%%%%%%%%%%%%%%%%%%%%%%%%%%%%%%%
% Extending the Network with Genes
To construct our \textbf{extended PPI network} we need to integrate genes into the created protein network.

First, we implement an index on the gene property $ID$ for faster querying.
We load the saved data (\ref{subsubsec:gene_nodes}~-~Gene nodes) as a list and create 17,626 gene nodes
in batches with their properties.

The cypher query employed for creating these gene nodes is:
\lstset{literate={Δ}{{\ensuremath{\Delta}}}1}
\begin{lstlisting}[language=Cypher, label={lst:gene_nodes}]
    CREATE (p:gene {
            id: 'Gene ID',
            gene_name: 'Gene Name',
            norm_healthy_tpm: 'norm healthy TPM',
            norm_cancerous_tpm: 'norm cancerous TPM',
            delta_tpm: 'Δ TPM',
            delta_type: 'Δ type',
            z_score: 'z score',
            delta_tpm_relevant: 'Δ TPM relevant'})
\end{lstlisting}

We characterize the genes in our network using these properties:
$gene\_name$, $norm\_healthy\_tpm$, $norm\_cancerous\_tpm$, $delta\_tpm$, $delta\_type$, $z\_score$, and $delta\_tpm\_relevant$.


Among these, the value of $delta\_tpm\_relevant$ stands out as a pivotal factor in our analysis.
Although other attributes may be less relevant to our current investigation, they retain potential value for future tasks.\\

To model relationships between genes and proteins,
we establish edges between these nodes by loading the gene-protein interaction data
(\ref{subsubsec:gene_protein_connections}~-~Gene-protein edges) as a list of tuples.
This involves matching gene and protein nodes based on their respective IDs,
and creating edges in batches to optimize processing efficiency.
The result is a comprehensive network of 101,731 connections between gene and protein nodes.

The cypher query for creating gene-protein connection edges is:
\begin{lstlisting}[language=Cypher, label={lst:gene_protein_edges}]
    MATCH (s:protein{id:'Protein ID'})
    MATCH (s:gene{id:'Gene ID'})
    CREATE (s)-[:connection]-(t)
\end{lstlisting}

% TODO Alternativ Verlinkung zu oben
\begin{figure}[h]
    \centering
    \includegraphics[width=1\textwidth]{figures/03_02_Network_2}
    \caption{Extract from the graph database setup showing example gene nodes and their connections to protein nodes}
    \label{fig:03_02_Network_2}
\end{figure}

Through the execution of these cypher queries,
we successfully populate our graph database with the required nodes and edges, enabling us to perform the PageRank algorithm.\\


\section{Results} \label{sec:results}
% focus on the results, not on the analysis


\section{Discussion} \label{sec:discussion}

% analysius of the results eg Top 10 Pagerank genes
\subsection{Analysis} \label{subsec:analysis}


% what went well / what went wrong
\subsection{What went well} \label{subsec:what-went-well}

% TODO discussion oder conclusion?
% what can be improved / future work
\subsection{Future Work} \label{subsec:future_work}
{\color{lightgray}
* Pagerank is a good way to find important genes - well established / known / typical research \\
* Clear finding of new genes that may be biomarkers\\
* Analyse the top 10 genes seems to find good results\\

% Future Work
* These should be further investigated in future studies\\
*  collective behavior of multiple genes rather than focusing solely on individual genes
}


\section{Conclusion} \label{sec:conclusion}

% Breif summary of the findings




\newpage
\bibliographystyle{plain}
% Style Types: abbrv, alpha, apalike, ieeetr, plain, unsrt
\bibliography{refs} % Entries are in the refs.bib file

\end{document}